\documentclass[12pt, a4paper]{article}

\usepackage[margin=1in]{geometry}
\usepackage[utf8]{inputenc}
\usepackage[spanish]{babel}
\usepackage{amssymb, amsmath, amsbsy}
\usepackage{mathpazo}
\usepackage{tasks}

\usepackage[pdftex]{hyperref}
\hypersetup{ pdftitle = {   },pdfauthor = {Luis Carrillo Gutiérrez} }
\renewcommand{\familydefault}{\sfdefault}

\begin{document}
\paragraph*{Problema} -- \\
\begin{itemize}
	\item{El producto de un número de dos cifras por el mismo pero con sus cifras en orden inverso es 574. Hallar la suma de sus cifras del número citado.
	\begin{tasks}(5)
		\task{8}
		\task{10}
		\task{15}
		\task{5}
		\task{11}
	\end{tasks}
	}
\end{itemize}
\paragraph*{Datos} -- \\
$$
\overline{ab} \cdot \overline{ba} = 574 
$$

\paragraph*{Solución} -- \\

\noindent Si $a = 1$, al  tabular como primera opción de número natural, entonces $b = 4$ (cifra necesaria para que el producto sea el número en unidades de $574$, en este caso $4$), por lo que $\overline{ab} = 14$, por ende $\overline{ba} = 41$, por consiguiente $\overline{ab} \cdot \overline{ba} = 574$ ó $14 \cdot 41 = 574$ \\

\begin{tabular}{cccc}
\empty & 1 & 4 & $\times$ \\
\empty & 4 & 1 & \empty \\
\hline 
\empty & 1 & 4 & $+$ \\
4 & 6 & \empty \\
\hline 
5 & 7 & 4 & \empty 
\end{tabular} \\ \\

$\therefore\quad\overline{ab}$ es $14$ y la suma de sus cifras es $\mathbf{5}$.

\end{document}