\documentclass[10pt, twocolumn, landscape, a4paper]{article}

\usepackage[margin=1in]{geometry}
\usepackage[utf8]{inputenc}
\usepackage[spanish]{babel}
\usepackage{amssymb, amsmath, amsbsy}
\usepackage{verbatim}
\usepackage{mathpazo}
\usepackage{tasks}
\usepackage{enumerate}
\usepackage[pdftex]{hyperref}
\hypersetup{ pdftitle = {   },pdfauthor = {Luis Carrillo Gutiérrez} }

\renewcommand{\familydefault}{\sfdefault}

\begin{document}
% UNSAAC - Grupo D - Tema Q
\subsection*{Educación Cívica}
\begin{itemize}
\item{La Soberanía del Estado Peruano sobre su espacio aéreo comprende :
\begin{tasks}
\task{El espacio que se alza en sus tres regiones naturales y mar territorial.}
\task{Solo el espacio atmosférico que ocupan las poblaciones asentadas.}
\task{El espacio que se alza sobre los cuarteles y bases militares de la república.}
\task{El espacio que se alza sobre los 305 aeropuertos y aeródromos que tiene el Perú.}
\task{Las rutas que recorren las aeronaves civiles y militares.}
\end{tasks}
}
\item{El Sistema Nacional de Defensa Civil está constituído por :
\begin{tasks}
\task{La población peruana en su conjunto.}
\task{Por los prefectos, alcaldes, jefes policiales y autoridades eclesiásticas.}
\task{Por el ejército, la marina y la aviación.}
\task{Organizaciones gremiales, barriales, compañía de bomberos y Boy's Scout's.}
\task{Por los ciudadanos mayores de 18 años de edad y reservistas.}
\end{tasks}
}
\item{De acuerdo a la Constitución del Estado Peruano las atribuciones del Congreso en materia Legislativa son :
\begin{tasks}
\task{Promulgar, derogar y vetar las leyes.}
\task{Derogar, modificar e interpretar las leyes existentes y aprobar otras nuevas.}
\task{Dar, derogar, interpretar y reglamentar.}
\task{Promulgar, interpretar, aplicar y derogar.}
\task{Analizar en comisiones, sancionarlas, concordarlas con la constitución y aplicar en las sentencias.}
\end{tasks}
}
\item{La Identidad Nacional se puede conceptuar como :
\begin{tasks}
\task{La integración histórica, étnica, cultural, idiomática y religiosa de los miembros de una Nación.}
\task{La comprensión y acercamiento dentro de un país de los jóvenes y personas adultas.}
\task{El reconocimiento de todos en una sociedad de los derechos del niño y la mujer.}
\task{La cohesión entre civiles, militares y religiosos de una sociedad.}
\task{Ninguno de los anteriores.}
\end{tasks}
}
\end{itemize}
\end{document}