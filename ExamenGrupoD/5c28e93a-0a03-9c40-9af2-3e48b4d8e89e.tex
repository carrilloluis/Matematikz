\documentclass[10pt, twocolumn, landscape, a4paper]{article}

\usepackage[margin=1in]{geometry}
\usepackage[utf8]{inputenc}
\usepackage[spanish]{babel}
\usepackage{amssymb, amsmath, amsbsy}
\usepackage{verbatim}
\usepackage{mathpazo}
\usepackage{tasks}
\usepackage{enumerate}
\usepackage[pdftex]{hyperref}
\hypersetup{ pdftitle = { },pdfauthor = {Luis Carrillo Gutiérrez} }

\renewcommand{\familydefault}{\sfdefault}

\begin{document}
% UNSAAC - Grupo D - Tema Q
\subsection*{Filosofía}
\begin{itemize}
\item{Los hombres son el resultado de un proceso de desarrollo histórico. ¿Qué queremos decir cuando afirmamos ésto?
\begin{tasks}
\task{El hombre tiene un comienzo fijo y un fin determinado.}
\task{El hombre como cualquier ser, es producto de la naturaleza.}
\task{Que el hombre es un ser que protagoniza la acción y modifica el mundo.}
\task{Los hombres son seres pensantes.}
\task{El hombre es un ser eminentemente ahistórico.}
\end{tasks}
}
\item{En el campo de la filosofía, la concepción idealista argumenta que los valores no nacen en el hombre ni en el mundo; sino, son esencias que trascienden lo natural. Esto significa :
\begin{tasks}
\task{Los verdaderos valores guían al hombre.}
\task{Los valores están presentes de por sí en las cosas.}
\task{La proyección del hombre es alcanzar lo positivo.}
\task{Que los valores existieron desde un principio, antes que todo.}
\task{Los valores no están en el hombre porque no es autosuficiente.}
\end{tasks}
}
\item{El acto del conocimiento, es todo un proceso que implica ingresar al campo mental en la aprehensión de una cualidad real de los fenómenos. En las siguientes afirmaciones hay un conocimiento logrado por vía racional, el cuál es :
\begin{tasks}
\task{Todos los animales toman agua.}
\task{Sólo cuando se toca o se palpa una cosa se pueden lograr conocimientos.}
\task{En plutón no existen signos de vida.}
\task{Los estudiantes conocen su responsabilidad.}
\task{El hombre es un ser social.}
\end{tasks}
}
\end{itemize}
\end{document}