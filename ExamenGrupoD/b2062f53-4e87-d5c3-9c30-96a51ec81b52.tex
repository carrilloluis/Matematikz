\documentclass[10pt, twocolumn, landscape, a4paper]{article}

\usepackage[margin=1in]{geometry}
\usepackage[utf8]{inputenc}
\usepackage[spanish]{babel}
\usepackage{amssymb, amsmath, amsbsy}
\usepackage{verbatim}
\usepackage{mathpazo}
\usepackage{tasks}
\usepackage{enumerate}
\usepackage[pdftex]{hyperref}
\hypersetup{ pdftitle = {   },pdfauthor = {Luis Carrillo Gutiérrez} }

\renewcommand{\familydefault}{\sfdefault}

\begin{document}
% UNSAAC - Grupo D - Tema Q
\subsection*{Historia del Perú}
\begin{itemize}
\item{La cultura Recuay, se desarrolló en :
\begin{tasks}(3)
\task{Ancash.}
\task{Ayacucho.}
\task{Cajamarca.}
\task{Trujillo.}
\task{Cusco.}
\end{tasks}
}
\item{¿Qué institución española dirigió todos los aspectos de administración colonial?
\begin{tasks}(2)
\task{El Cabildo.}
\task{La Intendencia.}
\task{El Virreynato.}
\task{El Corregimiento.}
\task{El Real Consejo de Indias.}
\end{tasks}
}
\item{La primera Junta de Gobierno que surgió en América (1809), fue en :
\begin{tasks}(3)
\task{Argentina.}
\task{Chile.}
\task{Alto Perú.}
\task{Quito.}
\task{Venezuela.}
\end{tasks}
}
\item{¿A qué escuadrón se debe el triunfo de la Batalla de Junín?
\begin{tasks}(2)
\task{División Lara.}
\task{Caballería Independiente.}
\task{División La Mar.}
\task{Húsares del Perú.}
\task{División Córdova.}
\end{tasks}
}
\item{Después de \textit{Augusto B. Leguía}, asumió la Presidencia de la República por elecciones generales de $1931$ :
\begin{tasks}(2)
\task{Luís M. Sánchez Cerro.}
\task{Oscar R. Benavides.}
\task{Manuel Prado.}
\task{José Luis Bustamante y Rivero.}
\task{Manuel A. Odría.}
\end{tasks}
}
\end{itemize}
\end{document}