\documentclass[10pt, twocolumn, landscape, a4paper]{article}

\usepackage[margin=1in]{geometry}
\usepackage[utf8]{inputenc}
\usepackage[spanish]{babel}
\usepackage{amssymb, amsmath, amsbsy}
\usepackage{verbatim}
\usepackage{mathpazo}
\usepackage{tasks}
\usepackage{enumerate}
\usepackage[pdftex]{hyperref}
\hypersetup{ pdftitle = {Sección de Álgebra del Examen para el grupo D en CEPRU},pdfauthor = {Luis Carrillo Gutiérrez} }

\renewcommand{\familydefault}{\sfdefault}

\begin{document}
% UNSAAC - Grupo D - Tema Q
\subsection*{Álgebra}

\begin{itemize}
\item{Al simplificar la expresión : $\quad x^{\dfrac{n \cdot \log(\log x)}{\log x}}$ se obtiene :
\begin{tasks}(5)
\task{$\log n$}
\task{$(\log x)^n$}
\task{$\log n^x$}
\task{$\log x^n$}
\task{$\log x$}
\end{tasks}
}
\item{Guido podría arribar a la casa de Donato distante $18$ Km, $1$ hora $30$ minutos antes que de costumbre si aumentase su velocidad en $1$ Km/h. ¿En qué tiempo Guido arribará a su destino?
\begin{tasks}(5)
\task{3 horas}
\task{8 horas}
\task{5 horas}
\task{4 horas}
\task{6 horas}
\end{tasks}
}
\item{Si la división $\dfrac{x^3 - ax - 6}{x^2 - bx - 6}$ es exacta, entonces la suma de los valores de $a$ y $b$ es : 

\begin{tasks}(5)
\task{-6}
\task{-8}
\task{8}
\task{7}
\task{6}
\end{tasks}
}
\item{Después de resolver : $\quad y^{\sqrt[4]{y}} = \sqrt[{\sqrt[4]{x}}]{\dfrac{x^{- \frac1{2}}}{4}}$ el valor de $xy$ es :
\begin{tasks}(5)
\task{$\dfrac1{4}$}
\task{$\dfrac1{16}$}
\task{$\dfrac1{8}$}
\task{$\dfrac1{64}$}
\task{$\dfrac1{32}$}
\end{tasks}
}
\item{El producto de las soluciones reales del sistema : 

\begin{eqnarray*}
	(5y - 2x)^2 &=& 41 - 20xy \\
	\Big(\dfrac{x + 3y}{2}\Big)^2 &=& \dfrac{13}{4} + \dfrac{3}{2}xy
\end{eqnarray*}
es : 
\begin{tasks}(5)
\task{$-4$}
\task{$4$}
\task{$\dfrac1{4}$}
\task{$-\dfrac1{4}$}
\task{$0$}
\end{tasks}
}
\item{La sexta parte de un camino es subida, la tercera parte es llano y el resto es bajada, una persona puede desarrollar en la subida $3$ Km/hr, en terreno plano $4$ Km/hr y en la bajada $4,5$ Km/hr invirtiendo en un viaje de ida y vuelta $1$ hora $48$ minutos. Durante la ida recorrió :
\begin{tasks}(5)
\task{$3{,}3$ Km}
\task{$4,2$ Km}
\task{$4,5$ Km}
\task{$3,6$ Km}
\task{$3,9$ Km}
\end{tasks}
}
\end{itemize}
\end{document}

