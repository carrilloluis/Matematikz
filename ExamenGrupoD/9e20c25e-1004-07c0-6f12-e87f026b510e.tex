\documentclass[10pt, twocolumn, landscape, a4paper]{article}

\usepackage[margin=1in]{geometry}
\usepackage[utf8]{inputenc}
\usepackage[spanish]{babel}
\usepackage{amssymb, amsmath, amsbsy}
\usepackage{verbatim}
\usepackage{mathpazo}
\usepackage{tasks}
\usepackage{enumerate}
\usepackage[pdftex]{hyperref}
\hypersetup{ pdftitle = {   },pdfauthor = {Luis Carrillo Gutiérrez} }

\renewcommand{\familydefault}{\sfdefault}

\begin{document}
% UNSAAC - Grupo D - Tema Q
\subsection*{Psicología}

\begin{itemize}
\item{Las reacciones fisiológicas que acompañan a las emociones y sentimientos tienen su origen anatómico principalmente en :
\begin{tasks}
\task{La médula espinal.}
\task{El corazón y los pulmones.}
\task{El hemisferio cerebral izquierdo.}
\task{El hipotálamo.}
\task{El bulbo raquídeo.}
\end{tasks}
}
\item{Cuando un niño aprende observando e imitando la conducta de los personajes de TV., está interviniendo el aprendizaje :
\begin{tasks}
\task{Por ensayo y error.}
\task{Por instrucción.}
\task{Por coacción y gratificación.}
\task{Por moderación.}
\task{Por semejanza y contigüedad.}
\end{tasks}
}
\item{El fenómeno psicológico que incita e impulsa al organismo a la acción dirigiéndola hacia determinados objetivos, se denomina : 
\begin{tasks}
\task{Comportamiento.}
\task{Motivación.}
\task{Afectividad.}
\task{Inteligencia.}
\task{Sensación.}
\end{tasks}
}
\item{¿A qué tipo de anomalía perceptiva corresponde la estructura de la siguiente imagen?
\begin{tasks}
\task{Alteración figura-fondo.}
\task{Ilusión.}
\task{Alucinación.}
\task{Astigmatismo.}
\task{Superposición líneal.}
\end{tasks}
}
\item{El proceso en el cual participan la familia, la escuela y la comunidad para formar roles, actitudes y destrezas que permitan la adaptación del individuo al grupo es :
\begin{tasks}
\task{El gregarismo.}
\task{La extroversión.}
\task{La socialización.}
\task{La asimilación.}
\task{La interacción.}
\end{tasks}
}
\item{Existen factores endógenos y exógenos que influyen integradamente en el proceso del desarrollo. ¿Cuál de los siguientes constituye un factor exógeno?
\begin{tasks}
\task{La herencia.}
\task{El crecimiento.}
\task{El aprendizaje.}
\task{La maduración.}
\task{La salud.}
\end{tasks}
}
\end{itemize}
\end{document}