\documentclass[10pt, twocolumn, landscape, a4paper]{article}

\usepackage[margin=1in]{geometry}
\usepackage[utf8]{inputenc}
\usepackage[spanish]{babel}
\usepackage{amssymb, amsmath, amsbsy}
\usepackage{verbatim}
\usepackage{mathpazo}
\usepackage{tasks}
\usepackage{enumerate}
\usepackage[pdftex]{hyperref}
\hypersetup{ pdftitle = {   },pdfauthor = {Luis Carrillo Gutiérrez} }

\renewcommand{\familydefault}{\sfdefault}

\begin{document}
% UNSAAC - Grupo D - Tema Q
\subsection*{Geografía del Perú y del Mundo}
\begin{itemize}
\item{Los andes del sur del Perú comprenden :
\begin{tasks}
\task{Las fronteras de Bolivia, Chile y la divisoria del Vilcabamba.}
\task{Las fronteras de Chile, Bolivia y el Valle del Vilcanota.}
\task{El océano Pacífico, Chile y la cadena del Vilcanota.}
\task{El océano Pacífico, Bolivia y la divisoria fluvial del Vilcanota.}
\task{Las fronteras con Chile, Bolivia y la divisoria fluvial del Vilcanota.}
\end{tasks}
}
\item{Un buen manejo de los recursos naturales renovables por parte del hombre hacen que puedan :
\begin{tasks}(3)
\task{Extinguirse.}
\task{Agotarse.}
\task{Reponerse.}
\task{Aniquilarse.}
\task{Depredarse.}
\end{tasks}
}
\item{Las estrellas del universo: Hidra y Orión pertenecen a las constelaciones :
\begin{tasks}(3)
\task{Boreales.}
\task{Zodiacales.}
\task{Australes.}
\task{Ecuatoriales.}
\task{Meridionales.}
\end{tasks}
}
\item{Las causas de las migraciones en el Perú son por :
\begin{tasks}
\task{Crecimiento desmedido de los centros urbanos.}
\task{La ruralización de las ciudades.}
\task{Mejores condiciones de vida en las ciudades que en el campo.}
\task{Crecimiento desmedido de los centros rurales.}
\task{Aumento de las necesidades en las ciudades.}
\end{tasks}
}
\item{¿En qué país de América del Sur está ubicado el pico del Chimborazo?
\begin{tasks}(3)
\task{Colombia.}
\task{Venezuela.}
\task{Perú}
\task{Bolivia.}
\task{Ecuador.}
\end{tasks}
}
\end{itemize}
\end{document}