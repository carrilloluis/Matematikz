\documentclass[10pt, twocolumn, landscape, a4paper]{article}

\usepackage[margin=1in]{geometry}
\usepackage[utf8]{inputenc}
\usepackage[spanish]{babel}
\usepackage{amssymb, amsmath, amsbsy}
\usepackage{verbatim}
\usepackage{mathpazo}
\usepackage{tasks}
\usepackage{enumerate}
\usepackage[pdftex]{hyperref}
\hypersetup{ pdftitle = {   },pdfauthor = {Luis Carrillo Gutiérrez} }

\renewcommand{\familydefault}{\sfdefault}

\begin{document}
% UNSAAC - Grupo D - Tema Q
\subsection*{ - }

\begin{itemize}
\item{Señale la oración que contiene la contracción del artículo:

\begin{tasks}
\task{Leí un libro muy bello.}
\task{Leí el diario ``El Sol" del Cusco.}
\task{Leí una noticia de ``El Sol".}
\task{Lo leí en el libro de Juan.}
\task{Leí la lección de mi libro.}
\end{tasks}
}
\item{Los signos de puntuación que indican las pausas son:
\begin{tasks}
\task{Corchete, paréntesis, asterisco, guión.}
\task{Interrogación, exclamación, paréntesis, guión.}
\task{Comillas, dos puntos, asterisco, diéresis.}
\task{Coma, punto y coma, puntos suspensivos, punto.}
\task{Punto seguido, punto aparte, interrogación, admiración.}
\end{tasks}
}
\end{itemize}
\end{document}