\documentclass[10pt, twocolumn, landscape, a4paper]{article}

\usepackage[margin=1in]{geometry}
\usepackage[utf8]{inputenc}
\usepackage[spanish]{babel}
\usepackage{amssymb, amsmath, amsbsy}
\usepackage{verbatim}
\usepackage{mathpazo}
\usepackage{tasks}
\usepackage{enumerate}
\usepackage[pdftex]{hyperref}
\hypersetup{ pdftitle = { },pdfauthor = {Luis Carrillo Gutiérrez} }

\renewcommand{\familydefault}{\sfdefault}

\begin{document}
% UNSAAC - Grupo D - Tema Q
\subsection*{Lógica}
\begin{itemize}
\item{¿Cuál de las afirmaciones nos muestra claramente que el lenguaje se usa en función expresiva?
\begin{tasks}
\task{El agua contiene dos moléculas de hidrógeno y una de oxígeno.}
\task{Quiero que modifiques tu conducta.}
\task{Qué desgracia la sucedida en Arequipa.}
\task{Cusco es una ciudad arqueológica muy interesante.}
\task{Los gnomos o duendes son seres imaginarios.}
\end{tasks}
}
\item{En las siguientes proposiciones, ¿Cuál es la proposición compuesta conjuntiva?. 
\begin{tasks}
\task{Las olimpiadas deportivas mundiales se efectúan cada cuatro años.}
\task{Si soy rico, soy feliz; no soy rico, entonces no soy feliz.}
\task{Si tuviera que decidir entre matemática o física no sabría que elección efectuar.}
\task{No podemos ser ángeles ni diablos al mismo tiempo.}
\task{Los terremotos y maremotos son desastres naturales.}
\end{tasks}
}
\item{¿Cuál es el resultado correcto de la siguiente proposición compuesta? : \\ $p \wedge q\,. \equiv\,.\,p \vee q$ (trabaje una tabla de verdad).
\begin{tasks}
\task{VFVF}
\task{FFFF}
\task{VVVV}
\task{FFFV}
\task{VVVF}
\end{tasks}
}
\end{itemize}
\end{document}