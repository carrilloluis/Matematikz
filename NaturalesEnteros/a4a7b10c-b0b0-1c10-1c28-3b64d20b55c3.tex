\documentclass[12pt, a4paper]{article}

\usepackage[margin=1in]{geometry}
\usepackage[utf8]{inputenc}
\usepackage[spanish]{babel}
\usepackage{amssymb, amsmath, amsbsy}
\usepackage{mathpazo}

\usepackage[pdftex]{hyperref}
\hypersetup{ pdftitle = {},pdfauthor = {Luis Carrillo Gutiérrez} }

\renewcommand{\familydefault}{\sfdefault}

\begin{document}

\paragraph*{Problema} -- \\

\noindent Si $\overline{mnu} + \overline{nmu} = \overline{yyy}$. Hallar $m + n$; si cada cifra ($m, n, u, y$) es un valor par distinto y pertenecen a los $\mathbb{N}$.

\paragraph*{Datos} -- \\

\begin{center}
\begin{tabular}{cccc}
	m & n & u & + \\
	n & m & u & \empty \\
	\hline
	y & y & y & \empty
\end{tabular}
\end{center}

\noindent Cifras que representan la suma en : \\
\hspace*{0.6cm}las unidades : $u + u = y \to 2u = y$ \\
\hspace*{0.6cm}las decenas : $n + m = y \wedge y = 2u \to m + n = 2u$ \\
\hspace*{0.6cm}las centenas : $m + n = y$ \\
Números pares en $\mathbb{N} : \{2, 4, 6, 8, \ldots \}$ 

\paragraph*{Solución} -- \\

\noindent Al tabular con $m$ en $2$ y $n$ en $4$, la suma $m + n$ es $6$, pero $u$ es $3$ ó $6 \div 2 = 3$, un número \textbf{impar}. \\
Al tabular con $m$ en $2$ y $n$ en $6$, la suma $m + n$ es $8$, y $u$ es $4$ ya que $8 \div 2 = 4$, un número \textbf{par},por tanto, cumpliendo además con la condición de ser \textit{números pares distintos}.
$$
\therefore m = 2,\,n = 6\quad y\quad u = 4
$$

\begin{center}
\begin{tabular}{cccc}
	2 & 6 & 4 & + \\
	6 & 2 & 4 & \empty \\
	\hline
	8 & 8 & 8 & \empty
\end{tabular}
\end{center}

$$
\therefore m  + n = 8
$$

\end{document}
