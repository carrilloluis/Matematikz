\documentclass[12pt, a4paper]{article}

\usepackage[margin=1in]{geometry}
\usepackage[utf8]{inputenc}
\usepackage[spanish]{babel}
\usepackage{amssymb, amsmath, amsbsy}
\usepackage{mathpazo}
\usepackage{tasks}

\usepackage[pdftex]{hyperref}
\hypersetup{ pdftitle = {   }, pdfauthor = {Luis Carrillo Gutiérrez} }
\renewcommand{\familydefault}{\sfdefault}

\begin{document}
\paragraph*{Problema} -- \\
\begin{itemize}
	\item{En una multiplicación, si al multiplicando se le agrega $5$ y al multiplicador se le quita $5$, entonces el producto aumenta en $600$. ¿Cuál es la diferencia del multiplicando y del multiplicador?.

	\begin{tasks}(5)
		\task{100}
		\task{515}
		\task{325}
		\task{645}
		\task{125}
	\end{tasks}
	}
\end{itemize}
\paragraph*{Datos} -- \\

$$
M \cdot m = p
$$

$$
(M + 5)\cdot (m - 5) = p + 600 
$$

\paragraph*{Solución} -- 
\begin{eqnarray*}
(M + 5) \cdot (m - 5) &=& p + 600  \\
M\cdot m - 5\cdot M + 5\cdot m - 25 &=& p + 600 \\
p - 5\cdot M + 5\cdot m &=& p + 600 + 25\\
p - 5 \cdot (M + m) &=& p + 625 \\
p + 5 \cdot (M - m) &=& p + 625 \\
5 \cdot (M - m) &=& 625 \\
M - m &=& \dfrac{625}{5} = 125 \\
\end{eqnarray*}

$$
\therefore\quad M - m = 125
$$

\end{document}