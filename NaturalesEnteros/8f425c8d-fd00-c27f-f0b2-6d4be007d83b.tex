\documentclass[12pt, a4paper]{article}

\usepackage[margin=1in]{geometry}
\usepackage[utf8]{inputenc}
\usepackage[spanish]{babel}
\usepackage{amssymb, amsmath, amsbsy}
\usepackage{enumerate}
\usepackage{mathpazo}
\usepackage{tasks}

\usepackage[pdftex]{hyperref}
\hypersetup{ pdftitle = {}, pdfauthor = {Luis Carrillo Gutiérrez} }
\renewcommand{\familydefault}{\sfdefault}

\begin{document}
% Sistema de Números naturales ($\mathbb{N}$) y Enteros ($\mathbb{Z}$)
\begin{itemize}
	\item{Relacionar correctamente las siguientes proposiciones :

\begin{enumerate}[I.]
\item El elemento neutro para la sustracción en los números naturales ($\mathbb{N}$) es el cero.
\item El orden de los factores no altera el producto, viene a ser la propiedad conmutativa de la adición.
\item El sistema de los números naturales ($\mathbb{N}$) no es denso.
\item El complemento aritmético del complemento aritmético de un número resulta el mismo número.

\end{enumerate}

	\begin{tasks}(5)
		\task{VFVF}
		\task{VFFV}
		\task{FVVV}
		\task{FFVF}
		\task{FFVV}
	\end{tasks}
	}
	\item{Relacionar correctamente las siguientes proposiciones :
	\begin{enumerate}[I.]
	\item El inverso aditivo de un número es el mismo número multiplicado por -1
	\item El uno es el elemento neutro de la multiplicación
	\item El residuo de una división algunas veces es menor que el divisor.
	\item En la división inexacta el residuo por defecto más el residuo por exceso es igual al divisor.
	\end{enumerate}

	\begin{tasks}(5)
		\task{VVVV}
		\task{VFFV}
		\task{VFVF}
		\task{VFVV}
		\task{VFFF}
	\end{tasks}
}

\end{itemize}

\end{document}