\documentclass[11pt, twocolumn, landscape, a4paper]{article}

\usepackage[margin=1.2cm]{geometry}
\usepackage[utf8]{inputenc}
\usepackage[spanish]{babel}
\usepackage{amssymb, amsmath, amsbsy}
\usepackage{tasks}
\usepackage{enumerate}
\usepackage{graphicx}

\usepackage[pdftex]{hyperref}
\hypersetup{pdftitle={Fórmulas en física},pdfauthor={Luis Carrillo Gutiérrez}}

\fontfamily{cmss}\selectfont
\renewcommand{\familydefault}{\sfdefault}
\pagestyle{empty}

\begin{document}
\section*{Física}

\paragraph*{Fuerza} % Force
$$
\sum F = \dfrac{d \mathbf{p}}{dt} = \dfrac{d (m \mathbf{v})}{dt}
$$

$$
\sum F = m\mathbf{a} % (constant mass) 
$$

\paragraph*{Velocidad} % Velocity
$$
\mathbf{v}_{promedio} = \dfrac{\Delta\,\mathbf{d}}{\Delta\,t} 
$$ % average

$$
\mathbf{v} = \dfrac{d\mathbf{s}}{dt}
$$

\paragraph*{Movimiento} % Motion
$$
v = v_{0} + at
$$

$$
\mathbf{s} = \dfrac1{2}(v_{0} + v)t
$$

$$
\mathbf{s} = v_{0}t + \dfrac1{2}at^{2}
$$

$$
v^{2} = v_{0}^{2} + 2a\mathbf{s} 
$$

\paragraph*{Aceleración} % Acceleration
$$
\mathbf{a}_{\,promedio} = \dfrac{\Delta\,\mathbf{v}}{\Delta\,t}
$$ % average

$$
\mathbf{a} = \dfrac{d\mathbf{v}}{dt} = \dfrac{d^{2}\mathbf{s}}{dt^{2}}
$$

\paragraph*{Energía cinética} % Kinetic Energy
$$
T = \dfrac1{2}m\mathbf{v}^{2}
$$

\paragraph*{Varianza} % Variance
$$
s^2 = \dfrac1{N} \sum_{i = 1}^{N} (x_{i} - \bar{x})^{2}
$$

\paragraph*{Gravedad} % Gravity
$$
F = \dfrac{Gm_{1}m_{2}}{r^{2}}
$$

\paragraph*{Torque} % Torque
$$
\sum \tau = \dfrac{d\mathbf{L}}{dt}
$$

$$
\sum \tau = r \times \mathbf{F}
$$

\paragraph*{Impulso} % Impulse
$$
\mathbf{J} = \Delta \mathbf{p} = \int \mathbf{F} dt
$$

$$
\mathbf{J} = \mathbf{F} \Delta t % if F is constant
$$

\paragraph*{Relación Masa-Energía} % Mass Energy
$$
E = mc^{2}
$$
\paragraph*{Ley de Drude} % Drude Law
$$
\alpha = \dfrac{k}{\lambda^{2} - \lambda_{0}^{2}}
$$

\paragraph*{Densidad} % Density
$$
\rho = \dfrac{m}{\mathcal{V}}
$$

\paragraph*{Carga} % Charge
$$
Q = It
$$
\end{document}