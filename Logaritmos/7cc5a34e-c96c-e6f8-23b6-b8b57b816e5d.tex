\documentclass[11pt, twocolumn, landscape, a4paper]{article}

\usepackage[margin=1.2cm]{geometry}
\usepackage[utf8]{inputenc}
\usepackage[spanish]{babel}
\usepackage{amssymb, amsmath, amsbsy}
\usepackage{tasks}
\usepackage{enumerate}
\usepackage{graphicx}

\usepackage[pdftex]{hyperref}
\hypersetup{pdftitle={Logaritmos},pdfauthor={Luis Carrillo Gutiérrez}}

\fontfamily{cmss}\selectfont
\renewcommand{\familydefault}{\sfdefault}
\pagestyle{empty}

\begin{document}
\subsection*{Logaritmos - II}
\begin{itemize}
\item{Resolver la siguiente ecuación : $\quad 3\cdot\log{x} - \log{32} = 2\cdot\log\Big(\dfrac{x}{2}\Big)$
\begin{tasks}(5)
\task{2}\task{8}\task{16}\task{24}\task{48}
\end{tasks}
}
\item{Hallar el valor de : $\quad 1 - co\log_{\,2} (antilog_{\,4} (\log_{\,5}{625}))$
\begin{tasks}(5)
\task{7}\task{8}\task{9}\task{10}\task{12}
\end{tasks}
}
\item{Si : $\log_{\,2}{4} + \log_{\,2}{4^{2}} + \ldots + \log_{\,2}{4^{n}} = \log_{\,2}{4^{6}}$. El valor de ``\scalebox{1.2}{$n$}'' es : 
\begin{tasks}(5)
\task{2}\task{3}\task{4}\task{5}\task{6}
\end{tasks}
}
\item{Hallar el valor de ``\scalebox{1.2}{$x$}'' en : $\quad \log_{\,8}{(25x - 7)^{3}} + co\log_{\,2}{(3x + 1)} = 3$
\begin{tasks}(5)
\task{6}\task{8}\task{12}\task{16}\task{N.A.}
\end{tasks}
}
\item{Hallar el valor de ``\scalebox{1.2}{$x$}'' en la ecuación: $\quad 2x \cdot \log_{4}{2} = \log^{2}_{\,4}{2} + 1$
\begin{tasks}(5)
\task{$\dfrac1{2}$}
\task{$\dfrac{4}{5}$}
\task{$\dfrac{5}{4}$}
\task{$2$}
\task{$5$}
\end{tasks}
}
\item{Calcular : $$2^{\log_{\,4}{9}} + 27^{\log_{\,3}{5}} + antilog_{\,2}{3} + co\log_{\,5}{0.2}$$
\begin{tasks}(5)
\task{126}
\task{135}
\task{137}
\task{140}
\task{221}
\end{tasks}
}
\item{Encontrar el mayor de dos números, uno el doble del otro, cuya diferencia es igual a la diferencia de sus logaritmos.
\begin{tasks}(5)
\task{$\log{2}$}\task{$\log{4}$}\task{$\log{3}$}\task{$100$}\task{$20$}
\end{tasks}
}
\item{Hallar ``\scalebox{1.2}{$x$}'' que satisface la ecuación : $\dfrac{27}{81^{- \log{x}}} = \dfrac1{3}$
\begin{tasks}(5)
\task{$10$}\task{$\dfrac1{10}$}\task{$1$}\task{$-10$}\task{$5$}
\end{tasks}
}
\item{Reducir la expresión de diez factores : $$\log_{\,2}{4} \cdot \log_{\,4}{8} \cdot \log_{\,8}{16} \cdot \ldots$$
\begin{tasks}(5)
\task{10}
\task{11}
\task{12}
\task{13}
\task{14}
\end{tasks}
}
\item{Calcular : $\log_{\,x}{10}$ a partir de la igualdad : $$\log_{\,x}\log{x} = \dfrac1{\log{0.2}x + \log_{\,0.1}{0.2}}$$
\begin{tasks}(5)
\task{$0.1$}
\task{$1$}
\task{$2$}
\task{$10$}
\task{$100$}
\end{tasks}
}
\item{Resolver : $e^{\ln{\,(2x - 1)}} + e^{\ln{\,x}} = 1$
\begin{tasks}(5)
\task{$\dfrac{3}{2}$}
\task{$2$}
\task{$5$}
\task{$6$}
\task{$\dfrac{2}{3}$}
\end{tasks}
}
\item{El equivalente de : $$\dfrac{(e^{x} + e^{y} + e^{z})(x + y + z)}{e^{\ln(z - y)} + e^{\ln(z - x)} + e^{\ln(x + y)}}$$\\ Siendo $x = \ln{3}$; $y = \ln{5}$; $z = \ln{15}$ es : 
\begin{tasks}(5)
\task{21}
\task{23}
\task{34}
\task{45}
\task{46}
\end{tasks}
}
\item{Si : $1 + \ln{3} = \ln{(\ln{(\ln{a})})}$. Calcular : $\sqrt[3]{\ln\sqrt[3]{\ln\sqrt[3]{\ln{a}}}}$
\begin{tasks}(5)
\task{$\sqrt[3]{3}$}
\task{$\sqrt[3]{\dfrac1{3}}$}
\task{$9$}
\task{$12$}
\task{$27$}
\end{tasks}
}
\end{itemize}
\end{document}