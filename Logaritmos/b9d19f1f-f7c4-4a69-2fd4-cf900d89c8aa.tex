\documentclass[11pt, twocolumn, landscape, a4paper]{article}

\usepackage[margin=1cm]{geometry}
\usepackage[utf8]{inputenc}
\usepackage[spanish]{babel}
\usepackage{amssymb, amsmath, amsbsy}
\usepackage{tasks}
\usepackage{enumerate}
\usepackage{graphicx}

\usepackage[pdftex]{hyperref}
\hypersetup{pdftitle={Conceptos en Logaritmos},pdfauthor={Luis Carrillo Gutiérrez}}

\fontfamily{cmss}\selectfont
\renewcommand{\familydefault}{\sfdefault}
\pagestyle{empty}

\begin{document}
\section*{Logaritmos}
\subsection*{Definición}
$$\log_{\,b}{N} = x\quad \implies \quad b^{x} = N$$
\subsection*{Propiedades}
\begin{enumerate}
\item{En el campo de los números reales ($\mathbb{R}$); no existe logaritmo de números negativos.}
\item{La base de un logaritmo debe ser siempre positiva y diferente de la unidad.}
\item{La identidad logarítmica fundamental se expresa como : \scalebox{1.2}{$b^{\,\log_{\,b}{N}} = N$}}
\item{El logaritmo de la unidad en cualquier base es cero : \scalebox{1.2}{$\log_{\,b}1 = 0$}}
\item{El logaritmo de la base será  siempre igual a la unidad : \scalebox{1.2}{$\log_{\,b}{b} = 1$}}
\item{Logaritmo de un producto : \scalebox{1.2}{$\log_{\,b}{a \cdot c} = \log_{\,b}{a} + \log_{\,b}{c}$}}
\item{Logaritmo de un cociente : \scalebox{1.2}{$\log_{\,b}{\Big(\dfrac{a}{c}\Big)} = \log_{\,b}{a} - \log_{\,b}{c}$}}
\item{Logaritmo de una potencia : \scalebox{1.2}{$\log_{\,b}{a^{c}} = c\cdot \log_{\,b}{a}$}}
\item{Logaritmo cuya base es una potencia : \scalebox{1.25}{$\log_{\,b^{c}}{a} = \frac1{c}\cdot \log_{\,b}{a}$}\\ $\implies$ \scalebox{1.25}{$\log_{\,a^{c}}{a^{d}} = \frac{d}{c}$}}
\item{Cambio de base de un logaritmo : \scalebox{1.2}{$\log_{\,b}{a} = \dfrac{\log_{\,c}{a}}{\log_{\,c}{b}}$}}
\end{enumerate}
\end{document}