\documentclass[10pt, twocolumn, landscape, a4paper]{article}

\usepackage[margin=1in]{geometry}
\usepackage[utf8]{inputenc}
\usepackage[spanish]{babel}
\usepackage{amssymb, amsmath, amsbsy}
\usepackage{verbatim}
\usepackage{mathpazo}
\usepackage{multicol}
\usepackage[pdftex]{hyperref}
\hypersetup{ pdftitle = {   },pdfauthor = {Luis Carrillo Gutiérrez} }

\renewcommand{\familydefault}{\sfdefault}

\begin{document}

\subsection*{Proporciones}

Es la igualdad de dos razones

$$
\dfrac{a}{b} = \dfrac{c}{d}
$$

\noindent $a$ y $d$ son los \textit{extremos} \\
$b$ y $c$ son los \textit{medios} \\

\subsubsection*{Leyes de las proporciones}

$$
\text{Si : }\qquad\dfrac{a}{b} = \dfrac{c}{d}\qquad\text{ entonces :}
$$
\begin{multicols}{2}
	\begin{itemize}
	\item{$ad = bc$}
	\item{$\dfrac{b}{a} = \dfrac{d}{c}$}
	\item{$\dfrac{a}{c} = \dfrac{b}{d}$}
	\item{$\dfrac{a + b}{b} = \dfrac{c + d}{d}$}
	\item{$\dfrac{a - b}{b} = \dfrac{c - d}{d}$}
	\item{$\dfrac{a + b}{a - b} = \dfrac{c + d}{c - d}$}
	\end{itemize}
\end{multicols}
\end{document}
