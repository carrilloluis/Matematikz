\item{La diferencia de $2$ números es $244$ y están en la relación de $7$ a $3$. ¿Cuál es el mayor de los números?
\begin{tasks}(5)
\task{427}
\task{356}
\task{429}
\task{359}
\task{431}
\end{tasks}
\noindent \textbf{Solución} -- 
\begin{eqnarray*}
	x - y &=& 244 \\
	\dfrac{x}{y} &=& \dfrac{7}{3} \to x = \dfrac{7y}{3} \\
	\dfrac{7y}{3} - \dfrac{3y}{3} &=& 244 \\
	\dfrac{4y}{3} &=& 244 \to y = \dfrac{244 \cdot 3}{4} \\
	y &=& 183 \to x - 183 = 244 \\
	x &=& 244 + 183 = \fbox{427}
\end{eqnarray*}
}

\item{Dos números están en la relación de 6 a 11 y la diferencia entre ellos es $20$. Hallar la suma de dichos números.
\begin{tasks}(5)
\task{65}
\task{68}
\task{63}
\task{60}
\task{48}
\end{tasks}
\noindent \textbf{Solución} -- 
\begin{eqnarray*}
a - b &=& 20 \\
\dfrac{a}{b} &=& \dfrac{11}{6} \to a = \dfrac{11 \cdot b}{6} \\ % permite no tener negativos, ni usar el valor absoluto
\dfrac{11 b}{6} - \dfrac{6y}{6} &=& 20  \\
\dfrac{5 b}{6} &=& 20 \\
b &=& \dfrac{20 \cdot 6}{5} = 24 \\
a &=& \dfrac{11 \cdot 24}{6} = 44 \\
\therefore\quad a + b &=& 44 + 24 = \fbox{68} 
\end{eqnarray*}
}

\item{En un colegio de $684$ alumnos, la razón entre el número de alumnos de primaria y secundaria es $47/10$, Determinar el número de alumnos de secundaria. 
\begin{tasks}(5)
\task{120}
\task{210}
\task{310}
\task{130}
\task{115}
\end{tasks}
\noindent \textbf{Solución} -- 
\begin{eqnarray*}
p + s &=& 684 \\
\dfrac{p}{s} &=& \dfrac{47}{10} \to p = \dfrac{47 \cdot s}{10} \\
\dfrac{47 \cdot s}{10} + \dfrac{10 \cdot s}{10} &=& 684 \\
\dfrac{57 \cdot s}{10} &=& 684 \\
s &=& \dfrac{10 \cdot 684}{57} = 10 \cdot 12 = \fbox{120} \\
\therefore\quad p &=& \dfrac{47 \cdot 120}{10} = 47 \cdot 12 = 564 
\end{eqnarray*}
}

\item{En un corral el número de patos excede al número de pavos en $75$. Además, se observa que por cada $8$ patos hay $5$ pavos. ¿Cuál es el número de patos y pavos que hay en el corral? 
\begin{tasks}(5)
\task{192}
\task{325}
\task{900}
\task{740}
\task{375}
\end{tasks}
\noindent \textbf{Solución} -- 
\begin{eqnarray*}
pt - pv &=& 75 \\
% 8 \cdot pt &=& 5 \cdot pv \\ 
\dfrac{pt}{8} &=& \dfrac{pv}{5 } \to pt = \dfrac{8 \cdot pv}{5} \\ 
\dfrac{8 \cdot pt}{5} - \dfrac{5 \cdot pv}{5} &=& 75 \\ 
\dfrac{3 \cdot pv}{5} &=& 75 \\ 
pv &=& \dfrac{75 \cdot 5}{3} = \dfrac{25 \cdot 3 \cdot 5}{3} = 125 \\ 
pt &=& \dfrac{8 \cdot pv}{5} = \dfrac{8 \cdot 125}{5} = \dfrac{8 \cdot 25 \cdot 5}{5} = 200 \\
\therefore\quad pv + pt &=& 125 + 200 = \fbox{325}
\end{eqnarray*}
}
