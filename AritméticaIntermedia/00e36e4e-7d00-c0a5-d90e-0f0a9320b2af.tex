\documentclass[11pt, twocolumn, landscape, a4paper]{article}

\usepackage[margin=1.2cm]{geometry}
\usepackage[utf8]{inputenc}
\usepackage[spanish]{babel}
\usepackage{amssymb, amsmath, amsbsy}
\usepackage{tasks}
\usepackage{enumerate}
\usepackage{graphicx}

\usepackage[pdftex]{hyperref}
\hypersetup{pdftitle={Aritmética intermedia},pdfauthor={Luis Carrillo Gutiérrez}}

\renewcommand{\familydefault}{\sfdefault}
\fontfamily{cmss}\selectfont
\pagestyle{empty}
 
\begin{document}
\subsection*{Aritmética intermedia}

\begin{itemize}
\item{La suma de los tres términos de una resta es seis veces el sustraendo. Si la diferencia es $42$. Hallar el minuendo.
\begin{tasks}(5)
\task{48}
\task{57}
\task{60}
\task{63}
\task{72}
\end{tasks}
}
\item{En el triángulo numérico. Hallar la suma de términos de la fila $30$.

\begin{tabular}{cccccccc}
$f_{1}$ & \empty & \empty & \empty & 2 & \empty & \empty & \empty \\
$f_{2}$ & \empty & \empty & 4 & \empty & 8 & \empty & \empty \\
$f_{3}$ & \empty & 8 & \empty & 10 & \empty & 12 & \empty \\
$f_{4}$ & 14 & \empty & 16 & \empty & 18 & \empty & 20 \\
\end{tabular}
\begin{tasks}(5)
\task{27010}
\task{27030}
\task{27050}
\task{27060}
\task{27090}
\end{tasks}
}
\item{La suma de dos números excede a la diferencia de los mismos en $256$. Si los números están en relación de $45$ a $20$. Hallar el mayor de ellos.
\begin{tasks}(5)
\task{260}
\task{270}
\task{273}
\task{279}
\task{288}
\end{tasks}
}
\item{Calcular : $S = 1 \times 7 + 2 \times 8 + 3 \times 9 + 4 \times 10 + \ldots + 20 \times 26$
\begin{tasks}(5)
\task{4150}
\task{4145}
\task{4140}
\task{4135}
\task{4130}
\end{tasks}
}
\item{En un campeonato que duró $39$ semanas, si en cada semana se jugó cuatro partidos. ¿Cuántos equipos participaron, sabiendo que han jugado de visita y de local?
\begin{tasks}(5)
\task{1270}
\task{13}
\task{14}
\task{15}
\task{16}
\end{tasks}
}
\item{Si ambas sumas tienen la misma cantidad de sumandos :
\begin{eqnarray*}
S_{1} &=& 40 + 41 + 42 + \ldots + n \\ 
S_{2} &=& 10 + 12 + 14 + \ldots + m
\end{eqnarray*}
y además $S_{1} = S_{2}$. Hallar ``$m + n$'' 
\begin{tasks}(5) 
\task{230} 
\task{240}
\task{250}
\task{260} 
\task{270}
\end{tasks}
}
\item{Hallar : $a + b + x + y\quad$; si $\quad\overline{a1b} + \overline{a2b} + \overline{a3b} + \ldots + \overline{a7b} = \overline{x8y1}$
\begin{tasks}(5) 
\task{11} 
\task{12} 
\task{13} 
\task{14} 
\task{15} 
\end{tasks}
}
\item{Calcular el valor de \scalebox{1.2}{$S$} si tiene $30$ sumandos : \\ $S = 3 + 100 + 6 + 98 + 9 + 96 + 12 + 94 + \ldots $
\begin{tasks}(5)
\task{1575} 
\task{1600} 
\task{1625}
\task{1650} 
\task{1675}
\end{tasks}
}
\item{Si la suma de los elementos de una sustracción es igual a $840$, en donde el minuendo es el triple del sustraendo. Hallar la suma de cifras de la diferencia.
\begin{tasks}(5)
\task{10}
\task{9}
\task{8}
\task{7}
\task{6}
\end{tasks}
}
\item{Si : \scalebox{1.1}{$\overline{xy} - \overline{yx} = \overline{a (b - 2)}$}. Hallar: \scalebox{1.1}{$\overline{ab} + \overline{ba}$}.
\begin{tasks}(5)
\task{64}
\task{81}
\task{100} 
\task{121} 
\task{144} 
\end{tasks}
}
\end{itemize}
\end{document}