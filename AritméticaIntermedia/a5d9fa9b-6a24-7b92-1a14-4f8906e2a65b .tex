\documentclass[11pt, twocolumn, landscape, a4paper]{article}

\usepackage[margin=1.2cm]{geometry}
\usepackage[utf8]{inputenc}
\usepackage[spanish]{babel}
\usepackage{amssymb, amsmath, amsbsy}
\usepackage{tasks}
\usepackage{enumerate}
\usepackage{graphicx}

\usepackage[pdftex]{hyperref}
\hypersetup{pdftitle={Aritmética intermedia},pdfauthor={Luis Carrillo Gutiérrez}}

\renewcommand{\familydefault}{\sfdefault}
\fontfamily{cmss}\selectfont
\pagestyle{empty}
 
\begin{document}
\subsection*{Aritmética intermedia}
% \subsection*{Problemas propuestos}
\begin{itemize}
\item{Si : \scalebox{1.15}{$\overline{3xy} + \overline{z4x} = \overline{ppp4}$}. Hallar : ``$x + y + x - p$''.
\begin{tasks}(5)
\task{33}
\task{31}
\task{24} 
\task{20}
\task{15}
\end{tasks}
}
\item{Si el {\tt C.A.} de : \scalebox{1.15}{$\overline{\raisebox{0.1ex}{pua}}$} es $\overline{(p + 3)(2u)(a - 2)}$. Hallar: ``$p + u + a$''.
\begin{tasks}(5)
\task{12} 
\task{13}
\task{14}
\task{15}
\task{17}
\end{tasks}
}
\item{Si el {\tt C.A.} de \scalebox{1.15}{$\overline{xyz} = \overline{bbb}$}, cuando $b < 3$. Hallar: ``$z\cdot b + x\cdot y$''. 
\begin{tasks}(5)
\task{65}  
\task{66}
\task{67}  
\task{68} 
\task{69} 
\end{tasks}
}
\item{Si : $\overline{x0y} - \overline{yzx} = \overline{abc}$ y $\overline{abc} - \overline{cba} = 99$. Hallar: ``$x - y$''.
\begin{tasks}(5)
\task{3}
\task{4}
\task{5}
\task{6}
\task{7}
\end{tasks}
}
\item{Si \,\scalebox{1.15}{$\overline{xyy} < 400$} y además $\quad\overline{xyy} = \overline{y} + \overline{xy} + \overline{zx0} + \overline{z0y}$. \\ Hallar ``$x + y + z$''.
\begin{tasks}(5)
\task{10}
\task{9}
\task{8}
\task{7}
\task{6}
\end{tasks}
}
\item{Hallar las tres últimas cifras de la suma : \\ 
$S = 6 + 64 + 643  + 6436 + \ldots + \underbrace{\,643643 \ldots\quad}_{63\quad cifras}$ 
\begin{tasks}(5)
\task{578}
\task{573}
\task{564}
\task{474}
\task{473}
\end{tasks}
}
\item{La suma de los números de dos cifras diferentes que se puede formar con tres cifras consecutivas a es igual a $396$. Hallar el mayor de los números.
\begin{tasks}(5)
\task{98}
\task{86}
\task{76}
\task{67}
\task{65}
\end{tasks}
}
\item{Hallar : $x\cdot y\cdot z$ en la siguiente adición : \\ \scalebox{1.25}{$\overline{zxzx} + \overline{8zyy} + \overline{y7xz} + \overline{xyz8} = \overline{24y22}$}
\begin{tasks}(5)
\task{72}
\task{84}
\task{90} 
\task{105}
\task{120}
\end{tasks}
}
\item{Hallar dos números de tres cifras cada una sabiendo que suman $1019$. Si se sabe que el {\tt C.A.} de uno de ellos es el doble del {\tt C.A.} del otro. Dar la suma de cifras del mayor.
\begin{tasks}(5)
\task{15}
\task{16}
\task{17}
\task{18}
\task{19}
\end{tasks}
}
\item{Si : $\overline{abc} + \overline{bca} + \overline{cab} = 2109$ y $\overline{abc} - \overline{bca} = 261$. Hallar: ``$a\cdot b\cdot c$''
\begin{tasks}(5)
\task{224}
\task{220}
\task{195}
\task{180} 
\task{172} 
\end{tasks}
}
\end{itemize}
\end{document}
