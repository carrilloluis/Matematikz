\documentclass[11pt, twocolumn, landscape, a4paper]{article}

\usepackage[margin=1.2cm]{geometry}
\usepackage[utf8]{inputenc}
\usepackage[spanish]{babel}
\usepackage{amssymb, amsmath, amsbsy}
\usepackage{tasks}
\usepackage{enumerate}
\usepackage{graphicx}

\usepackage[pdftex]{hyperref}
\hypersetup{pdftitle={Aritmética de numerales},pdfauthor={Luis Carrillo Gutiérrez}}

\fontfamily{cmss}\selectfont
\renewcommand{\familydefault}{\sfdefault}
\pagestyle{empty}

\begin{document}
\subsection*{Aritmética}
 
\begin{itemize}
\item{Determine : $\quad a + b + c$. \quad Si $\quad\overline{abc}_{(5)} = 155_{(7)}$
\begin{tasks}(5)
\task{6}
\task{7}
\task{8}
\task{9}
\task{10}
\end{tasks}
}
\item{Determine : $\quad a + b + n$. \quad Si $\quad\overline{2a4b}_{(n)} = \overline{1131}_{(6)}$
\begin{tasks}(5)
\task{4}
\task{6}
\task{7}
\task{8}
\task{9}
\end{tasks}
}
\item{Determine : $\quad a + b + c + d + n$. \quad Si $\quad\overline{abcd}_{(n)} = \overline{101}_{(3)}$
\begin{tasks}(5)
\task{3}
\task{4}
\task{5}
\task{6}
\task{7}
\end{tasks}
}
\item{Determine : $\quad a + b + c$. \quad Si $\quad\overline{abc}_{(5)} = \overline{c07}_{(8)}$
\begin{tasks}(5)
\task{4}
\task{6}
\task{7}
\task{8}
\task{10}
\end{tasks}
}
\item{Determine : $\quad a + b + c$. \quad Si $\quad\overline{abc}_{(5)} = \overline{31a}_{(4)}$
\begin{tasks}(5)
\task{5}
\task{6}
\task{7}
\task{8}
\task{9}
\end{tasks}
}
\item{¿Cuántas de las siguientes proposiciones son verdaderas?
\begin{enumerate}[I.]
\item{Si un numeral esta expresado en diferentes sistemas, entonces se cumple que {\tt a menor base le corresponde mayor numeral}.}
\item{La cifra máxima disponible en sistema de base $15$ es $13$.}
\item{En todo sistema de numeración la base es mayor que las cifras de un numeral.}
\item{En cualquier sistema de numeración la cantidad de cifras posibles a utilizar siempre será numéricamente igual a la base.}
\item{La máxima cifra que se puede escribir en cualquier sistema de numeración siempre será igual a la base disminuída en una unidad.}
\end{enumerate}
\begin{tasks}(5)
\task{1}\task{2}\task{3}\task{4}\task{5}
\end{tasks}
}
\item{Hallar : $\quad n + a$. \quad Si\quad \scalebox{1.15}{$\overline{nnn}_{(a)} = 4210$}.
\begin{tasks}(5)
\task{20}\task{25}\task{30}\task{35}\task{36}
\end{tasks} 
}
\vfill\null
\item{¿Cuántas de las siguientes proposiciones son falsas?
\begin{enumerate}[I.] 
\item{En todo sistema de numeración la base es siempre mayor o igual que las cifras del numeral.}
\item{Si la base es $20$, entonces el sistema es vigesimal.}
\item{En una sistema de numeración de base ``$n$'' se dispone de ``$n - 1$'' cifras para representar a todos los numerales.}
\item{Existen tres numerales de la forma: $\overline{a(a + 1)(a + 2)}_{(6)}$.}
\end{enumerate}
\begin{tasks}(5)
\task{0}\task{1}\task{2}\task{3}\task{4}
\end{tasks}
}
\item{A un numeral de dos cifras se le suma el que resulta al invertir el orden de sus cifras, se obtiene $187$. ¿Cuál es el producto de las cifras del número?
\begin{tasks}(5)
\task{24}\task{30}\task{45}\task{56}\task{72}
\end{tasks}
}
\item{De las siguientes proposiciones :
\begin{enumerate}[I.]
\item{La menor base conocida es unitaria.}
\item{El menor número de dos cifras es $99$.}
\item{El mayor número de tres cifras de la base senaria es $555_{(6)}$}
\item{Se sabe que : $\alpha = 11$ y $\beta = 12$.}
\item{En la base ternaria se puede emplear las cifras: $0; 1; 2; 3$.} 
\end{enumerate}
¿Cuántas proposiciones son falsas?
\begin{tasks}(5)
\task{1}\task{2}\task{3}\task{4}\task{5}
\end{tasks}
}
\end{itemize}
\end{document}
