\documentclass[11pt, twocolumn, landscape, a4paper]{article}

\usepackage[margin=1cm]{geometry}
\usepackage[utf8]{inputenc}
\usepackage[spanish]{babel}
\usepackage{amssymb, amsmath, amsbsy}
\usepackage{tasks}
\usepackage{enumerate}
\usepackage{graphicx}

\usepackage[pdftex]{hyperref}
\hypersetup{pdftitle={Aritmética de numerales},pdfauthor={Luis Carrillo Gutiérrez}}

\fontfamily{cmss}\selectfont
\renewcommand{\familydefault}{\sfdefault}
\pagestyle{empty}

\begin{document}
\subsection*{Aritmética - IV / Numerales}

\begin{itemize}
\item{Expresar :\quad$\overline{(n - 4)(2m - 1)(n + 1)(m - 3)}_{(7)}$\quad en el sistema quinario. Dar como respuesta la suma de las cifras (en decimales).
\begin{tasks}(5)
\task{1}\task{2}\task{6}\task{8}\task{12}
\end{tasks}
}
\item{Hallar : \quad$a + x$. \quad Si : $\overline{a0aa0a}_{(x)} = 3315$
\begin{tasks}(5)
\task{2}\task{3}\task{5}\task{7}\task{9}
\end{tasks}
}
\item{Hallar el $a\cdot b + c\cdot d + e$ máximo.\quad Si $\overline{abcde}_{(5)} = \overline{460(e + 4)}_{(7)}$
\begin{tasks}(5)
\task{12}\task{14}\task{16}\task{21}\task{41}
\end{tasks}
}
\item{Cuál es el menor valor de : $a + b + c$.\quad Si $122_{(a)} = 101_{(b)} = 72_{(c)}$
\begin{tasks}(5)
\task{26}\task{25}\task{24}\task{23}\task{22}
\end{tasks}
}
\item{Determinar la suma de las cifras del siguiente número capicúa : $\overline{(3a - 1)(2b)(3c)\Big(\dfrac{b}{3} + 4\Big)}_{(9)}$
\begin{tasks}(5)
\task{16}\task{15}\task{11}\task{9}\task{8}
\end{tasks}
}
\end{itemize}
\end{document}