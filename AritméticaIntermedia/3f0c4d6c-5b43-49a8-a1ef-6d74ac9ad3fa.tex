\documentclass[11pt, twocolumn, landscape, a4paper]{article}

\usepackage[margin=1.2cm]{geometry}
\usepackage[utf8]{inputenc}
\usepackage[spanish]{babel}
\usepackage{amssymb, amsmath, amsbsy}
\usepackage{tasks}
\usepackage{enumerate}
\usepackage{graphicx}

\usepackage[pdftex]{hyperref}
\hypersetup{pdftitle={Aritmética de numerales},pdfauthor={Luis Carrillo Gutiérrez}}

\fontfamily{cmss}\selectfont
\renewcommand{\familydefault}{\sfdefault}
\pagestyle{empty}

\begin{document}
\subsection*{Aritmética - III / Numerales}

\begin{itemize}
\item{Al multiplicar un número de dos cifras por tres se obtiene el mismo resultado que al multiplicar por ocho al número que se obtiene al invertir el orden de sus cifras ¿Cuál es dicho resultado?
\begin{tasks}(5)
\task{216}\task{144}\task{72}\task{36}\task{27}
\end{tasks}
}
\item{Si a un numeral de dos cifras se le agrega el triple de la suma de sus cifras se obtiene $42$. Determine la suma de las cifras del número.
\begin{tasks}(5)
\task{9}\task{7}\task{6}\task{5}\task{3}
\end{tasks}
}
\item{Un número de tres cifras, es igual a tres veces el número formado por sus dos primeras cifras pero en orden inverso. Determine la suma de las cifras del número inicial.
\begin{tasks}(5)
\task{13}\task{11}\task{9}\task{8}\task{6}
\end{tasks}
}
\item{Un número aumentado en el triple de su cifra de decenas resulta $93$. Hallar la suma de sus cifras.
\begin{tasks}(5)
\task{11}\task{9}\task{8}\task{7}\task{6}
\end{tasks}
}
\item{¿Cuántos números de dos cifras son iguales a siete veces la suma de sus cifras?
\begin{tasks}(5)
\task{1}\task{2}\task{3}\task{4}\task{5}
\end{tasks}
}
\end{itemize}
\end{document}