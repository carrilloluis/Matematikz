\documentclass[11pt, twocolumn, landscape, a4paper]{article}

\usepackage[margin=1cm]{geometry}
\usepackage[utf8]{inputenc}
\usepackage[spanish]{babel}
\usepackage{amssymb, amsmath, amsbsy}
\usepackage{tasks}
\usepackage{enumerate}
\usepackage{graphicx}

\usepackage[pdftex]{hyperref}
\hypersetup{pdftitle={Aritmética de numerales},pdfauthor={Luis Carrillo Gutiérrez}}

\fontfamily{cmss}\selectfont
\renewcommand{\familydefault}{\sfdefault}
\pagestyle{empty}

\begin{document}
\subsection*{Aritmética - III / Numerales}

\begin{itemize}
\item{Al multiplicar un número de dos cifras por tres se obtiene el mismo resultado que al multiplicar por ocho al número que se obtiene al invertir el orden de sus cifras ¿Cuál es dicho resultado?
\begin{tasks}(5)
\task{216}\task{144}\task{72}\task{36}\task{27}
\end{tasks}
}
\item{Si a un numeral de dos cifras se le agrega el triple de la suma de sus cifras se obtiene $42$. Determine la suma de las cifras del número.
\begin{tasks}(5)
\task{9}\task{7}\task{6}\task{5}\task{3}
\end{tasks}
}
\item{Un número de tres cifras, es igual a tres veces el número formado por sus dos primeras cifras pero en orden inverso. Determine la suma de las cifras del número inicial.
\begin{tasks}(5)
\task{13}\task{11}\task{9}\task{8}\task{6}
\end{tasks}
}
\item{Un número aumentado en el triple de su cifra de decenas resulta $93$. Hallar la suma de sus cifras.
\begin{tasks}(5)
\task{11}\task{9}\task{8}\task{7}\task{6}
\end{tasks}
}
\item{¿Cuántos números de dos cifras son iguales a siete veces la suma de sus cifras?
\begin{tasks}(5)
\task{1}\task{2}\task{3}\task{4}\task{5}
\end{tasks}
}
\item{La suma de las cifras de un número es $14$ y si al número se suma $36$, las cifras se invierten. Dar como respuesta la diferencia de las cifras de dicho número de dos cifras.
\begin{tasks}(5)
\task{5}\task{4}\task{3}\task{2}\task{1}
\end{tasks}
}
\item{Un número está compuesto de tres cifras. La cifra de las centenas es cuatro veces la cifra de las unidades y la cifra de las decenas es igual a la mitad de la suma de las otras cifras. Dar como respuesta el producto de dichas cifras.
\begin{tasks}(5)
\task{90}\task{80}\task{64}\task{48}\task{36}
\end{tasks}
}
\item{Si a un número de tres cifras se le agrega un $5$ al comienzo y otro $5$ al final, el número obtenido es $147$ veces el número original. Dar como respuesta la suma de las cifras del número original.
\begin{tasks}(5)
\task{10}\task{11}\task{12}\task{13}\task{14}
\end{tasks}
}
\item{Un número está comprendido entre $100$ y $300$, es tal que leído al revés excede en $50$ al doble del número que le sigue al original. Hallar la suma de las cifras del número original.
\begin{tasks}(5)
\task{9}\task{10}\task{11}\task{12}\task{15}
\end{tasks}
}
\item{El número $\overline{a76b}$ es igual a $388$ veces la suma de sus cifras, Hallar el valor de : $a + b$.
\begin{tasks}(5)
\task{11}\task{9}\task{7}\task{6}\task{4}
\end{tasks}
}
\item{Determinar : $a + b + c$.\quad  Si \quad$\overline{abc}_{(5)} = 97$
\begin{tasks}(5)
\task{15}\task{11}\task{10}\task{9}\task{8}
\end{tasks}
}
\item{Hallar : $m + n + z$.\quad  Si \quad$\overline{a(a-3)(2a + 1)(a + 2)}_{(8)} = \overline{\ldots mnz}_{(3)}$
\begin{tasks}(5)
\task{1}\task{2}\task{3}\task{4}\task{5}
\end{tasks}
}
\item{Determinar : $a + b + n$.\quad Si \quad$\overline{abab}_{(n)} = 555$
\begin{tasks}(5)
\task{9}\task{10}\task{11}\task{12}\task{13}
\end{tasks}
}
\item{Hallar la suma de todos los valores de : $a + b + c + d$. \quad Si \quad$\overline{abcd} = 25\cdot\overline{ab} + 36\cdot\overline{cd}$
\begin{tasks}(5)
\task{71}\task{57}\task{54}\task{40}\task{17}
\end{tasks}
}
\item{Hallar $x + y + n$.\quad Si \quad$\overline{xyxy}_{(n)} = 1450$
\begin{tasks}(5)
\task{23}\task{17}\task{15}\task{12}\task{8}
\end{tasks}
}
\end{itemize}
\end{document}
