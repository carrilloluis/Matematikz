\documentclass[10pt, twocolumn, landscape, a4paper]{article}

\usepackage[margin=1in]{geometry}
\usepackage[utf8]{inputenc}
\usepackage[spanish]{babel}
\usepackage{amssymb, amsmath, amsbsy}
\usepackage{verbatim}
\usepackage{mathpazo}
\usepackage{enumerate}
% \usepackage{tikz}
% \usetikzlibrary{angles,quotes}
\usepackage[pdftex]{hyperref}
\hypersetup{ pdftitle = {   },pdfauthor = {Luis Carrillo Gutiérrez} }

\renewcommand{\familydefault}{\sfdefault}

\begin{document}
\begin{itemize}
\subsection*{Geometría Analítica}
% Introducción al Análisis Matemático.
% Capítulo 4 - Página 182
\item{Si $\vec{a} = (-3,\,5)$, $\vec{b} = (2,\, -3)$, Hallar la longitud del vector $\vec{c}$ para :
\begin{enumerate}[i. ]
\item $\vec{c} = (\vec{a} + \vec{b}) \cdot (\vec{a} - 2\vec{b})\;\vec{b}^{\,\perp}$
\item $\vec{c} = (\vec{a} \cdot \vec{b})\;\vec{b}^{\,\perp} - (\vec{a}^{\,\perp} \cdot \vec{b})\;\vec{c}$
\end{enumerate}}

\item{Pruebe que en cualquier paralelogramo la suma de los cuadrados de las longitudes de las diagonales es igual al doble de la suma de los cuadrados de dos lados adyacentes.}

\item{Dados los vectores $\vec{a}$, $\vec{b}$ y $\vec{p}$ en $\mathbb{R}^2$ determinar los números $r$ y $s$ en términos de sus productos escalares tal que el vector $\vec{p} - r\vec{a} - s\vec{b}$ sea perpendicular a los vectores $\vec{a}$ y $\vec{b}$ simultáneamente.}

\item{Sea $ABC$ un triángulo y $H$ la intersección de las alturas que pasan por $A$ y $B$. Probar que la tercera altura también pasa por $H$; es decir $\vec{HC} \perp \vec{AB}$.}

\item{ Sea $G = \dfrac1{3} (A + B + C)$ un punto del triángulo $ABC$ y sea $P$ cualquier punto, demostrar que : $$||\overrightarrow{PA}||^2 + ||\overrightarrow{PB}||^2 + ||\overrightarrow{PC}||^2 = 3 ||\overrightarrow{PG}||^2 + ||\overrightarrow{GA}||^2 + ||\overrightarrow{GB}||^2 + ||\overrightarrow{GC}||^2$$}

\item{Si $\vec{a}$ y $\vec{b}$ son vectores en $\mathbb{R}^2$ tales que $\vec{a}^{\,\perp} + \vec{b}^{\,\perp} = \vec{a} + \vec{b}$, demostrar que $||\vec{a}|| = ||\vec{b}||$}

\item{Si $\vec{a}$ y $\vec{b}$  son vectores paralelos no nulos, y $\vec{a} = (12, 5)$ es tal que $||\vec{a} + \vec{b}|| = ||a|| ||b||$, Hallar $a + b$ }
\end{itemize}

\end{document}
