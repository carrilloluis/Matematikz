\documentclass[10pt, twocolumn, landscape, a4paper]{article}

\usepackage[margin=1in]{geometry}
\usepackage[utf8]{inputenc}
\usepackage[spanish]{babel}
\usepackage{amssymb, amsmath, amsbsy}
\usepackage{verbatim}
\usepackage{mathpazo}
\usepackage{tasks}
\usepackage{enumerate}
\usepackage[pdftex]{hyperref}
\hypersetup{ pdftitle = {   },pdfauthor = {Luis Carrillo Gutiérrez} }

\renewcommand{\familydefault}{\sfdefault}

\begin{document}
% UNSAAC - Grupo D - Tema Q
\subsection*{Aritmética}
\begin{itemize}
\item{El valor de : $E = \sqrt{\dfrac{5 \cdot 4^{\,n - 1} }{2^{\,2n - 2} + 4^{\,n - 2}}}$ 

\begin{tasks}(5)
\task{$2$}
\task{$\sqrt{2}$}
\task{$\sqrt{3}$}
\task{$3$}
\task{$2\sqrt{2}$}
\end{tasks}
}
\item{Un vehículo que costó $USD\$\,6250$ se vende en los $\dfrac{3}{5}$ de su costo ¿Cuánto se pierde?
\begin{tasks}(5)
\task{$2750$}
\task{$1500$}
\task{$2000$}
\task{$2500$}
\task{$3750$}
\end{tasks}

}
\item{Si se tiene la siguiente tabla de valores para dos magnitudes $A$ y $B$ \\

\begin{tabular}{|c|c|c|c|c|c|}
	\hline 
	$A$ & 9 & 4 & 36 & 144 & 324 \\
	\hline
	$B$ & 12 & 18 & 6 & 3 & 2 \\
	\hline 
\end{tabular} \\

Entonces:

\begin{tasks}
\task{$A$ es proporcional a $B^3$}
\task{$A$ es proporcional a $\dfrac1{B^2}$}
\task{$A$ es proporcional a $\dfrac1{B}$}
\task{$A$ es proporcional a $B$}
\task{$A$ es proporcional a $B^2$}
\end{tasks}
}
\item{Para $a$ y $b$ números enteros, se tiene que : $a = -b$, entonces 
\begin{enumerate}[I) ]
\item{$b$ es el elemento neutro aditivo.}
\item{$b$ es el elemento inverso aditivo de $a$.}
\item{$a$ es el elemento inverso aditivo de $b$.}
\end{enumerate}
son verdaderos : 

\begin{tasks}(5)
\task{I y III}
\task{Sólo I}
\task{II y III}
\task{Sólo II}
\task{Sólo III}
\end{tasks}
}
\item{La media aritmética de $20$ números es $40$, cuando se considera un número más, la media aritmética disminuye en una unidad. El número considerado es :
\begin{tasks}(5)
\task{19}
\task{21}
\task{39}
\task{20}
\task{18}
\end{tasks}
}
\item{Si el valor de una letra es los $\dfrac{3}{4}$ de su valor nominal ¿A qué tasa de porcentaje se hizo el descuento?
\begin{tasks}(5)
\task{$30\%$}
\task{$25\%$}
\task{$20\%$}
\task{$15\%$}
\task{$10\%$}
\end{tasks}
}
\end{itemize}
\end{document}