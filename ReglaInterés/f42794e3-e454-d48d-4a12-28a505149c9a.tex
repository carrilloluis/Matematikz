\documentclass[11pt, twocolumn, landscape, a4paper]{article}

\usepackage[margin=1cm]{geometry}
\usepackage[utf8]{inputenc}
\usepackage[spanish]{babel}
\usepackage{amssymb, amsmath, amsbsy}
\usepackage{tasks}
\usepackage{enumerate}
\usepackage{graphicx}

\usepackage[pdftex]{hyperref}
\hypersetup{pdftitle={Regla de interés},pdfauthor={Luis Carrillo Gutiérrez}}

\fontfamily{cmss}\selectfont
\renewcommand{\familydefault}{\sfdefault}
\pagestyle{empty}

\begin{document}
\subsection*{Regla de interés}

\begin{itemize}
\item{Un capital estuvo impuesto al 9\% de interés anual. Si se obtuvo después de cuatro años $10\;200$\texttt{um.} ¿Cuál es el valor del capital?
\begin{tasks}(5)
\task{2500}\task{2700}\task{5700}\task{7200}\task{7500}
\end{tasks}
}
\item{¿Qué interés produce un capital de $3\;240$\texttt{um.} impuesto al 12\% durante cinco años?
\begin{tasks}(5)
\task{1944}\task{2944}\task{3944}\task{4944}\task{5944}
\end{tasks}
}
\item{¿Cuál es el capital que colocado al 7,5\% semestral durante ocho meses produce $24,3$\texttt{um.} de interés simple.
\begin{tasks}(5)
\task{640}\task{523}\task{496}\task{260}\task{243}
\end{tasks}
}
\item{¿A qué tanto por ciento habrá estado prestado un capital de $1200$\texttt{um.} para que se origine un beneficio de $1240$\texttt{um.} en veinte meses?
\begin{tasks}(5)
\task{10\%}\task{12\%}\task{14\%}\task{16\%}\task{20\%}
\end{tasks}
}
\item{¿Qué interés producirá un capital de $5200$\texttt{um.}, prestado al 7\% cuatrimestral en siete años y cinco meses?
\begin{tasks}(5)
\task{10099}\task{8099}\task{8090}\task{6418}\task{6410}
\end{tasks}
}
\end{itemize}
\end{document}