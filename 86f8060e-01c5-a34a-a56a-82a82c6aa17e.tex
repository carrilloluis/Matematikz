\documentclass[11pt, twocolumn, landscape, a4paper]{memoir}

\usepackage[margin=.8in]{geometry}
\usepackage[utf8]{inputenc}
\usepackage[spanish]{babel}
\usepackage{amssymb, amsmath, amsbsy}
\usepackage{verbatim}
\usepackage{mathpazo}
\usepackage{tasks}
\usepackage{enumerate}
\usepackage{graphicx} % \scalebox{1.4}{$$}
% \usepackage{tikz}
% \usetikzlibrary{angles,quotes}
\usepackage[pdftex]{hyperref}
\hypersetup{ pdftitle = {Ecuaciones },pdfauthor = {Luis Carrillo Gutiérrez} }

\renewcommand{\familydefault}{\sfdefault}

\begin{document}
% Ejercicios propuestos
\subsection*{Ecuaciones}
\begin{itemize}
\item{Hallar el valor de ``\scalebox{1.2}{$x$}'' en : $\dfrac1{2}\Big[\dfrac1{2}\Big[\dfrac1{2}\Big[\dfrac{x}{2} - 1 \Big] - 1 \Big] - 1 \Big] = 0$
\begin{tasks}(5)
\task{26}
\task{28}
\task{30}
\task{32}
\task{50}
\end{tasks}
}

\item{Resolver : $(x + 1) + (x + 2) + (x + 3) + \ldots + (x + n) = n^2$
\begin{tasks}(5)
\task{$\dfrac{n}{2}$}
\task{$\dfrac{n + 1}{2}$}
\task{$\dfrac{n - 1}{2}$}
\task{$n + 1$}
\task{$\dfrac{n + 3}{2}$}
\end{tasks}
}

\item{Resolver : $\overline{2x + 3} + \overline{3x + 2} - \overline{2x + 5} = \overline{3x}$
\begin{tasks}(5)
\task{1}
\task{2}
\task{3}
\task{4}
\task{5}
\end{tasks}
}

\item{Hallar el valor de ``\scalebox{1.2}{$x$}'' en : $\overline{x - 20 + 10\,\overline{x - 5}} = 8$
\begin{tasks}(5)
\task{14}
\task{12}
\task{20}
\task{5}
\task{10}
\end{tasks}
}

\item{Hallar el valor de ``\scalebox{1.2}{$x$}'' en : $$\overline{x + 44 + 14\,(x - 5)} + \overline{x + 59 + 8\,(4x - 20)} = 19$$ 
\begin{tasks}(5)
\task{9}
\task{7}
\task{4}
\task{5}
\task{6}
\end{tasks}
}

\item{Si la ecuación : $\dfrac{px + 1}{px - 1} + \dfrac{px - 1}{px + 1} = \dfrac{x + p}{px - 1}$, se reduce a una de primer grado; resolver e indicar su raíz.
\begin{tasks}(5)
\task{$\dfrac{7}{6}$}
\task{$\dfrac{6}{5}$}
\task{$\dfrac{3}{5}$}
\task{$\dfrac{1}{5}$}
\task{$\dfrac{8}{9}$}
\end{tasks}
}
\item{Resolver : $$
\begin{tasks}
\task{}
\end{tasks}
}
\item{Hallar n de modo que tenga raíces reales e iguales
\begin{tasks}
\task{}
\end{tasks}
}
\item{Hallar el producto de las raíces de la décima ecuación
\begin{tasks}
\task{}
\end{tasks}
}
\item{Si $x_1 \wedge x_2$ son raíces de la ecuación Calcular
\begin{tasks}
\task{}
\end{tasks}
}
\end{itemize}
\end{document}