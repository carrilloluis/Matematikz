\documentclass[11pt, twocolumn, landscape, a4paper]{article}

\usepackage[margin=1in]{geometry}
\usepackage[utf8]{inputenc}
\usepackage[spanish]{babel}
\usepackage{amssymb, amsmath, amsbsy}
\usepackage{verbatim}
\usepackage{mathpazo}
\usepackage{tasks}
\usepackage{enumerate}
\usepackage[pdftex]{hyperref}
\hypersetup{ pdftitle = { Introducción a la trigonometría },pdfauthor = {Luis Carrillo Gutiérrez} }

\renewcommand{\familydefault}{\sfdefault}

\begin{document}
\section*{Formulario de trigonometría}

\paragraph*{Función} -- Es el conjunto de pares ordenados de números $(x,\, y)$ tales que dos parejas distintas no tienen el mismo primer término.

\paragraph*{Dominio} -- Conjunto de todos los posibles valores de ``$x$''.
\paragraph*{Rango} -- Conjunto de todos los posibles valores de ``$y$''.
\paragraph*{Gráfica de una función} -- En un plano cartesiano, es el conjunto de todos los puntos $(x,\, y)$ que pertenecen a la función.
\paragraph*{Notación} -- Se denota : $$ y = f(x)\qquad\qquad\text{(Léase ``$f$'' de ``$x$'')}$$
\paragraph*{Funciones especiales} --
\begin{itemize}
\item{Función identidad : $$y = x$$}
\item{Función líneal o línea recta : $$y = mx + b$$}
\item{Función cuadrática o parábola : $$y = ax^2 + bx + c$$}
\item{Ecuación de circunferencia de centro en el origen del SCR : $$x^2 + y^2 = r^2$$}
\end{itemize}
\paragraph*{Funciones transcendentes} -- Son aquellas no algebraicas tales como las trigonométricas, logarítmicas o exponenciales.

\end{document}
