\documentclass[11pt, a5paper]{memoir}

\usepackage[margin=.75in]{geometry}
\usepackage[utf8]{inputenc}
\usepackage[spanish]{babel}
\usepackage{amssymb, amsmath, amsbsy}
\usepackage{mathpazo}
\usepackage{tasks}
% \usepackage{enumerate} \usepackage{verbatim}
% \usepackage{graphicx} % \scalebox{1.4}{$$}
\usepackage{tikz}
\usetikzlibrary{angles,quotes}
\usepackage[pdftex]{hyperref}
\hypersetup{pdftitle={Ángulos y triángulos},pdfauthor = {Luis Carrillo Gutiérrez}}

\renewcommand{\familydefault}{\sfdefault}

\begin{document}

\begin{itemize}
\item{En la siguiente figura el lado $\overline{AC}$ es bisectriz del ángulo $\angle BAD$. Determina los ángulos interiores de los triángulos $ABC$ y $ACD$ sabiendo que $\angle BAC = y + 8^{\circ}$, $\angle CAD = x + 13^{\circ}$, \\ $\angle ABC = 3x - 6^{\circ}\quad$ y $\quad\angle ACD = \dfrac{10}{3}y + 7^{\circ}$.

\begin{tikzpicture}
	\coordinate (A) at (.75, 2.1);
	\coordinate (B) at (0, 0);
	\coordinate (C) at (1.5, 0);
	\coordinate (D) at (4, 0);
	\coordinate (E) at (5.5, 0);
	\draw (B) -- (A) -- (D) -- cycle;
	\draw (D) -- (E);
	\draw (A) -- (C);
	
	\draw (-.25, -.25) node{$B$};
	\draw (.75, 2.35) node{$A$};
	\draw (1.5, -.25) node{$C$};
	\draw (4, -.25) node{$D$};
	\draw (5.5, -.25) node{$E$};

	\draw pic[draw, angle radius=.25cm] {angle=E--D--A};
	\draw (4.4, .5) node{$145^{\circ}$};
\end{tikzpicture}
% \begin{tasks}\task{}\end{tasks}
}
\end{itemize}
\end{document}