\documentclass[12pt, a5paper]{memoir}

\usepackage{geometry}
\usepackage[utf8]{inputenc}
\usepackage{amssymb, amsmath, amsbsy}
\usepackage{mathpazo}
\usepackage[pdftex]{hyperref}
\hypersetup{pdftitle={Valor Numérico en Polinomios},pdfauthor = {Luis Carrillo Gutiérrez} }

\renewcommand{\familydefault}{\sfdefault}

\begin{document}
\subsection*{Valor Numérico en Polinomios}
\begin{enumerate}
\item{Si $\mathrm{P}(x - 3) = 5x - 7$ ; $\mathrm{P}(\mathrm{F}(x) + 2) = 10x - 17$ ; Hallar el valor numérico de $\mathrm{F}(x - 2)$. \\
{\centering {\scriptsize Solución}}
\begin{eqnarray*}
P(x - 3) & = & 5x - 7 \\
P(x - 3 + 3) & = & 5(x + 3) - 7 \\
P(x) & = & 5x + 15 - 7 \\
P(x) & = & 5x + 8 \\
\empty \\
P(F(x) + 2) & = & 10x - 17 \\
P(F(x) + 2 - 2) & = & 10(x - 2) - 17 \\
P(F(x)) & = & 10x - 20 - 17 \\
5(F(x)) + 8 & = & 10x - 37 \\
F(x) & = & \dfrac{10x - 37 - 8}{5} \\
F(x) & = & \dfrac{10x - 45}{5} \\
F(x) & = & 2x - 9 \\
\empty \\
F(x - 2) & = & 2(x - 2) - 9 \\
F(x - 2) & = & 2x - 4 - 9 \\
F(x - 2) & = & 2x - 13 
\end{eqnarray*} \\
}
\item{
Si $\mathrm{F}(x -2) = 3x + 4$ ; Además si $\mathrm{F}(\mathrm{G}(x)) = 12x + 64$ ; Hallar $\mathrm{G}(-4)$.\\
{\scriptsize Solución}
\begin{eqnarray*}
F(x -2) & = & 3x + 4 \\
F(x -2 + 2) & = & 3(x + 2) + 4 \\
F(x) & = & 3x + 6 + 4 \\
F(x) & = & = 3x + 10 \\
\empty \\
F(G(x)) & = & 3(G(x)) + 10 \\
12x + 64 & = & 3(G(x)) + 10 \\
3(G(x)) + 10  & = & 12x + 64\\
G(x) & = & \dfrac{12x + 64 - 10}{3} \\
G(x) & = & \dfrac{12x + 54}{3} \\
G(x) & = & \dfrac{12x}{3} + \dfrac{54}{3} \\
G(x) & = & 4x + 18 \\
\empty \\
G(-4) & = & 4(-4) + 18 \\
G(-4) & = & -16 + 18 \\
G(-4) & = & 2 \\
\end{eqnarray*} \\
}
\item{Si $\mathrm{P}(7x -3)= 7x -1$ ;  $\mathrm{P}(\mathrm{F}(x) + 4) = 4x - 11$ ; Hallar el valor de $\mathrm{F}(x - 1)$.\\
{\scriptsize Solución}
\begin{eqnarray*}
P(7x - 3) & = & 7x - 1 \\
P(7x - 3 + 3) & = & 7(x + 3) - 1 \\
P(7x) & = & 7x + 21 - 1 \\
P(7 \dfrac{x}{7}) & = & 7(\dfrac{x}{7}) + 20 \\
P(x) & = & x + 20 \\
\empty \\
P(F(x) + 4) & = & 4x - 11 \\
P(F(x) + 4 - 4) & = & 4(x - 4) - 11 \\
P(F(x)) & = & 4x - 16 - 11 \\
P(F(x)) & = &  4x - 27 \\
\empty \\
P(F(x)) & = & F(x) + 20 \\
4x - 27 & = & F(x) + 20 \\
4x - 27 - 20 & = & F(x) \\
F(x) & = & 4x - 47 \\
F(x - 1) & = & 4(x - 1) - 47 \\
F(x - 1) & = & 4x - 4 - 47 \\
F(x - 1) & = & 4x - 51 \\
\end{eqnarray*} \\
}
\item{ Si $\mathrm{P}(x) = 3x^2 - 2x - 1$ ; Hallar el valor de :\\ \indent $\qquad\dfrac{P(0)^{P(1)} + P(1)^{P(2)}}{P(2) \cdot P(2)^{P(0)}}$ \\
{\scriptsize Solución}
\begin{eqnarray*}
P(x) & = & 3x^2 - 2x - 1 \\
P(0) & = & 3(0)^2 - 2(0) - 1 \\
P(0) & = & - 1 \\
\empty \\
P(1) & = & 3(1)^2 - 2(1) - 1 \\
P(1) & = & 3(1) - 2(1) - 1 \\
P(1) & = & 3 - 2 - 1 \\
P(1) & = & 0 \\
\empty \\
P(2) & = & 3(2)^2 - 2(2) - 1 \\
P(2) & = & 3(4) - 2(2) - 1 \\
P(2) & = & 12 - 4 - 1 \\
P(2) & = & 7 \\
& = & \dfrac{P(0)^{P(1)} + P(1)^{P(2)}}{P(2) \cdot P(2)^{P(0)}} \\
& = & \dfrac{(-1)^0 + 0^7}{7 \cdot 7^{-1}} = \dfrac{1 + 0}{1} \\
& = & 1
\end{eqnarray*} \\
}
\item{ Si $\mathrm{F}(F(x)) = 81x + 20$ ; $\mathrm{F}(Q(x) + 3) = 4x - 2$ ; Hallar $Q(10)$. \\
{\scriptsize Solución}
\begin{eqnarray*}
F(Q(x) + 3) & = & 4x - 2 \\
F(Q(x) + 3 - 3) & = & 4(x - 3) - 2 \\
F(Q(x)) & = & 4x - 12 - 2 = 4x - 14 \\
\empty \\
F(Q(x)) & = & 4x - 14 \\
F(F(x)) = 81x + 20
\end{eqnarray*} \\
}
\end{enumerate}
\end{document}