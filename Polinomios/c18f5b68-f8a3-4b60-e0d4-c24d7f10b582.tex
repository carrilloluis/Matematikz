\documentclass[11pt, a5paper]{memoir}

\usepackage{geometry}
\usepackage[utf8]{inputenc}
\usepackage[spanish]{babel}
\usepackage{enumerate}
\usepackage{amsmath, amssymb, amsbsy}
\usepackage{mathpazo}
\usepackage[pdftex]{hyperref}
\hypersetup{pdftitle={Polinomios},pdfauthor={Luis Carrillo Gutiérrez}}

\renewcommand{\familydefault}{\sfdefault}
\DeclareMathSizes{14}{14}{14}{14}

\begin{document}
\section*{Polinomios I}
\subsection*{Expresión algebraica}
\noindent Es una expresión matemática en la cual, para la variable o variables sólo se definen las operaciones aritméticas (adición, sustracción, multiplicación, división, potencia y raíz) un número finito de veces.
% Ejemplo:
\begin{description}
\item[] $R(x) = 6x - 5$
\item[] $S(x;y) = 29x^3 - \sqrt[7]{xy}$
\item[] $Q(x) = 1 + x + x^2 + x^3 + \ldots$
\end{description}

\subsection*{Término algebraico}
\noindent Es una expresión algebraica que no admite las operaciones de adición y sustracción.
% Ejemplo:
\begin{description}
\item $Q(x;y) = 5x^3y^5$
\item $R(x;y) = \dfrac{3x^3 \sqrt{y}}{2xy}$
\end{description}

\begin{enumerate}[A. ]
\item{Partes de un término algebraico.
$$
P(x;y) = 5 \pi x^8 y^5
$$
}
\item{Términos semejantes \\ Son aquellos términos que tienen la misma parte literal (las mismas variables afectados por los mismos exponentes).
% Ejemplo:
$$
R(x;\,y) = \sqrt{2} x^3y^5 \wedge Q(x;\,y) = 5x^3y^5
$$
}
\end{enumerate}

\subsection*{Monomio}
\noindent Es un término algebraico, cuyos exponentes de sus variables son números naturales.
% Ejemplo
\begin{description}
\item $R(x; y) = 2x^5y^4$
\item $Q(x; y) = \sqrt{3} xy^4$
\end{description}

\subsection*{Polinomio}
Es aquella expresión algebraica cuyos exponentes de sus variables son enteros no negativos (positivos o cero)
% Ejemplo
\begin{description}
\item{$P(x;y) = 3x^2y^5 - 2x^3y^2 - 8$ es un polinomio}
\item{$R(x;y) = \pi x^3y^{}$ no es un polinomio}
\item{$Q(x;y) = 2x^{-3}y^4 - 3x^2y^5$ no es polinomio}
\end{description}

\subsection*{Grados de un polinomio}
Los grados se clasifican en:
\begin{enumerate}
\item{Grado Relativo (G.R) | Es el mayor exponente de la variable de referencia}
\item{Grado Absoluto (G.A) | Se define como el grado de un polinomio. $$P() = 3x^5y^4z - 3x^2y^3z^2 + 3xy^6$$}
\end{enumerate}

\subsection*{Valor numérico}
Es el resultado de cambiar la variable por una constante.
% Ejemplo
Sea $P(x) = 3x + 2$; Calcula $M = P(2) + P(0)$

\subsection*{Valor numérico}

\subsection*{Cambio de variable}
Consiste en cambiar una variable por otra.
% Ejemplo
Sea $P(x) = 2x - 3$, calcula $P(3x - 5)$

\end{document}