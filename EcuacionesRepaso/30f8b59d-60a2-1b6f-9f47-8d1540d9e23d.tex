\documentclass[11pt, twocolumn, landscape, a4paper]{memoir}

\usepackage[margin=.75in]{geometry}
\usepackage[utf8]{inputenc}
\usepackage[spanish]{babel}
\usepackage{amssymb, amsmath, amsbsy}
\usepackage{verbatim}
\usepackage{mathpazo}
\usepackage{tasks}
\usepackage{enumerate}
\usepackage{graphicx} % \scalebox{1.4}{$$}
% \usepackage{tikz}
% \usetikzlibrary{angles,quotes}
\usepackage[pdftex]{hyperref}
\hypersetup{pdftitle={Ecuaciones - Repaso  II},pdfauthor={Luis Carrillo Gutiérrez}}

\renewcommand{\familydefault}{\sfdefault}

\begin{document}
\subsection*{Ecuaciones}
\begin{itemize}
\item{Determinar ``\scalebox{1.2}{$m$}'' y ``\scalebox{1.2}{$n$}'' tales que las ecuaciones : \\ $(5m - 52)\,x^2 - (m - 4)\,x + 4 = 0$\,; \\ $(2n + 1)\,x^2 - (5n)\,x + 20 = 0$; tengan las mismas raíces.
\begin{tasks}(3)
\task{9 y 7}
\task{11 y 7}
\task{7 y 8}
\task{10 y 9}
\task{12 y 8}
\end{tasks}
}
\item{¿Qué valor debe tomar ``$a$'' para que la ecuación : $\dfrac{a}{b} (x - a) =  \dfrac{b}{a} (x - b)$\,; sea incompatible ($a \neq b$)?
\begin{tasks}(5)
\task{$b$}
\task{$3b$}
\task{$-b$}
\task{$5b$}
\task{$2b$}
\end{tasks}
}
\item{En la ecuación : $2x^2 - (m - 1)\,x + (m + 1) = 0$\,; que valor positivo debe darse a ``$m$'' para que las raíces difieran en 1.
\begin{tasks}(5)
\task{1}
\task{11}
\task{10}
\task{4}
\task{5}
\end{tasks}
}
\item{Hallar el mayor valor de ``$m$'' si : $x^2 - (3m - 2)\,x + m^2 = 1$\,; cumple que otra raíz es el triple de la otra.
\begin{tasks}(5)
\task{1}
\task{2}
\task{3}
\task{4}
\task{8}
\end{tasks}
}
\item{ Hallar el conjunto de valores de ``$k$'' para el cuál la ecuación : $kx^2 + 8x + 4 = 0$\,; no tenga raíces reales.
\begin{tasks}(3)
\task{$[4, +\infty>$}
\task{$[2, +\infty>$}
\task{$8, +\infty$}
\task{$<-\infty, 4$}
\task{$4, +\infty$}
\end{tasks}
}
\item{Resolver : \scalebox{1.2}{$\sqrt[3]{2 - x} - \sqrt[3]{1 - x} = 1$}
\begin{tasks}(5)
\task{$1;-2$}
\task{$1;2$}
\task{$-1;2$}
\task{$2;3$}
\task{$1;3$}
\end{tasks}
}
\item{Hallar el valor de ``\scalebox{1.1}{$k$}'' en la ecuación : \\ $5k^2x^4 - 4k^4x^2 + 3k^2 = -2x^4 - 9x^2 - 6$\,; si el producto de sus raíces es 1.
\begin{tasks}(5)
\task{$\sqrt{2}$}
\task{$-1$}
\task{$3$}
\task{$\sqrt{3}$}
\task{$1$}
\end{tasks}
}
\item{Hallar ``\scalebox{1.1}{$x$}'' e ``\scalebox{1.1}{$y$}'' en el sistema : \\ $\dfrac1{x} + \dfrac{3}{y + 1} = \dfrac{5}{4}$  \\ $\dfrac{4}{x} - \dfrac{7}{y + 1} = \dfrac1{4}$
\begin{tasks}(5)
\task{1;2}
\task{2;1}
\task{1;3}
\task{3;2}
\task{2;3}
\end{tasks}
}
\item{Calcular el menor valor del producto de ``\scalebox{1.1}{$x$}'' e ``\scalebox{1.1}{$y$}'' en el \\ siguiente sistema de ecuaciones : \\ $x + \sqrt{x + y} = 32$ \\ $y + \sqrt{x + y} = 31$
\begin{tasks}(3)
\task{100}
\task{200}
\task{450}
\task{600}
\task{500}
\end{tasks}
}
\item{Calcular ``$m$'' entero si la ecuación : \\ $x^2 - 2\,(m - 1)x + 4m -7 = 0$; \\ tiene raíces complejas.
\begin{tasks}(5)
\task{-2, 2}
\task{$4, \infty$}
\task{2, 4}
\task{2, 4}
\task{2, 4}
\end{tasks}
}
\end{itemize}
\end{document}