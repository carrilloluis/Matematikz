\documentclass[11pt, a5paper]{article}

\usepackage[margin=.5in]{geometry}
\usepackage[utf8]{inputenc}
\usepackage[spanish]{babel}
\usepackage{amssymb, amsmath, amsbsy}
\usepackage{mathpazo}
\usepackage{tasks}
\usepackage{graphicx}
\usepackage[pdftex]{hyperref}
\hypersetup{pdftitle={Ecuaciones},pdfauthor={Luis Carrillo Gutiérrez}}

\renewcommand{\familydefault}{\sfdefault}
\pagestyle{empty}

\begin{document}
\subsection*{Ecuaciones}
\begin{itemize}
\item{Si una raíz de la ecuación : $5(x + a) = x + b + 6$ es $-2$. Calcular el valor de : $M = \dfrac{b + 9}{a - 1}$
\begin{tasks}(5)
\task{1}
\task{2}
\task{3}
\task{4}
\task{5}
\end{tasks}
}
\item{Resolver : $$\sqrt{21 + \sqrt{12 + \sqrt{14 + \sqrt{x}}}} = 5$$
\begin{tasks}(5)
\task{4/3}
\task{3}
\task{4}
\task{16}
\task{9}
\end{tasks}
}
\item{Resolver : $8x + 2(x + 1) = 7(x - 2) + 3(x + 1) + 13$
\begin{tasks}(3)
\task{5}
\task{{\small Infinitas\\ soluciones}}
\task{8}
\task{23}
\task{-6}
\end{tasks}

}
\item{Resolver : $$\dfrac{x}{1 \times 2} + \dfrac{x}{2 \times 3} + \dfrac{x}{3 \times 4} + \ldots + \dfrac{x}{99 \times 100} = \dfrac{198}{25}$$ Dar respuesta al valor de : $\sqrt[3]{x}$, Se sugiere usar : $\dfrac1{a(a + 1)} = \dfrac1{a} - \dfrac1{a + 1}$ 
\begin{tasks}(5)
\task{-1}
\task{1}
\task{0}
\task{2}
\task{3}
\end{tasks}
}
\item{Calcular ``\scalebox{1.1}{$mn$}'' si la ecuación : $\quad\dfrac{mx + 3}{x + 1} = \dfrac{n}{2}\quad$ es compatible indeterminada. 
\begin{tasks}(5)
\task{12}
\task{18}
\task{72}
\task{54}
\task{45}
\end{tasks}
}
\end{itemize}
\end{document}