\documentclass[11pt, a5paper]{memoir}

\usepackage[margin=.75in]{geometry}
\usepackage[utf8]{inputenc}
\usepackage[spanish]{babel}
\usepackage{amssymb, amsmath, amsbsy}
% \usepackage{verbatim}
\usepackage{mathpazo}
\usepackage{tasks}
% \usepackage{enumerate}
\usepackage{graphicx} 
\usepackage{tikz}
\usetikzlibrary{angles,quotes}
\usepackage[pdftex]{hyperref}
\hypersetup{pdftitle={Paralelepípedo Rectangular},pdfauthor={Luis Carrillo Gutiérrez}}

\renewcommand{\familydefault}{\sfdefault}
\pagenumbering{roman}

\begin{document}
\subsection*{Geometría - poliedros}
\begin{itemize}
\item{Calcular ``\scalebox{1.2}{$x$}'' (en el siguiente paralelepípedo rectangular): \\ \\
\begin{tikzpicture}
	\draw (0, 0) -- (4, 0) -- (4, 4) -- (0, 4) -- cycle;
	\draw (0, 4) -- (1, 4.75) -- (5, 4.75) -- (4, 4);
	\draw (5, 4.75) -- (5, 1) -- (4, 0);
	\draw [dashed] (1, 4.75) -- (1, 1);
	\draw [dashed] (0,0) -- (1, 1) -- (5, 1);
	\draw [line width=.5mm] (1, 4.75) -- (4, 0);

	\draw (2.1, -.35) node{$4\;\,cm$};
	\draw (5.2, .45) node{$3\;\,cm$};
	\draw (5.6, 2.75) node{$5\;\,cm$};

	\draw (2.8, 2.5) node{$x$};
\end{tikzpicture}

\begin{tasks}(3)
\task{$\dfrac1{2}\sqrt{2}$}
\task{$\dfrac1{3}\sqrt{2}$}
\task{$3\sqrt{2}$}
\task{$4\sqrt{2}$}
\task{$5\sqrt{2}$}
\end{tasks}
}
\end{itemize}

\end{document}