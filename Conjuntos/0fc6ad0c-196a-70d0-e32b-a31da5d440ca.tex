\documentclass[10pt, twocolumn, landscape, a4paper]{article}

\usepackage[margin=1in]{geometry}
\usepackage[utf8]{inputenc}
\usepackage[spanish]{babel}
\usepackage{amssymb, amsmath, amsbsy}
\usepackage{verbatim}
\usepackage{mathpazo}
\usepackage{tasks}
\usepackage{enumerate}
% \usepackage{tikz}
% \usetikzlibrary{angles,quotes}
\usepackage[pdftex]{hyperref}
\hypersetup{ pdftitle = { Práctica de Conjuntos },pdfauthor = {Luis Carrillo Gutiérrez} }

\renewcommand{\familydefault}{\sfdefault}

\begin{document}
\subsection*{Conjuntos}
\begin{itemize}
\item{Dado el conjunto $A = \{4;\,3;\,\{6\};\,8\}$ y las proposiciones : 
\begin{itemize}
\begin{minipage}{0.2\linewidth}
\item{$\{3\} \in A$}
\item{$\{6\} \in A$}
\item{$8 \in A$}
\item{$\emptyset \in A$}
\end{minipage}
\begin{minipage}{0.2\linewidth}
\item{$\{4\} \subset A$}
\item{$\{6\} \subset A$}
\item{$\varnothing \subset A$}
\item{$\{3;\,8\} \subset A$}
\end{minipage}
\end{itemize}
 
Indique el número de proposiciones verdaderas :
\begin{tasks}(5)
	\task{7}
	\task{6}
	\task{5}
	\task{4}
	\task{3}
\end{tasks}}
\item{Dados los conjuntos iguales $A = \{ a^2 + 3;\,b + 1\}$ y $B = \{13;\,19\}$. Considere $a$ y $b$ como enteros. Indique la suma de los valores que toma: $a + b$
\begin{tasks}(5)
	\task{16}
	\task{24}
	\task{30}
	\task{12}
	\task{27}
\end{tasks}}
\item{Indique la suma de los elementos del conjunto : \\ $R = \{x^2 + 2 / x \in \mathbb{Z}^+ \wedge -4 < x < 4\}$
\begin{tasks}(5)
	\task{23}
	\task{24}
	\task{20}
	\task{22}
	\task{25}
\end{tasks}}
\item{¿Cuántos subconjuntos propios tiene el siguiente conjunto
: \\ $G = \{2;\,3;\,\{2\};\,3;\,2;\,\{2\};\,\{3\}\}$ ?
\begin{tasks}(5)
	\task{127}
	\task{63}
	\task{15}
	\task{7}
	\task{31}
\end{tasks}}
\item{Si : $A = \{x/x = (4m -1)^2; m \in \mathbb{N} \wedge 2 \leq m \leq 5\}$ \\ Entonces el conjunto $A$ escrito por extensión es :
\begin{tasks}(3)
	\task{$\{7;\,11;\,15;\,19\}$}
	\task{$\{2;\,3;\,4;\,5\}$}
	\task{$\{4;\,9;\,16;\,25\}$}
	\task{$\{49;\,121;\,225;\,361\}$}
	\task{$\{3;\,4;\,7;\,9\}$}
\end{tasks}}
\item{En un grupo de $100$ estudiantes, $49$ no llevan el curso de Sociología y $53$ no siguen el curso de Filosofía. Si $27$ alumnos no siguen Filosofía ni Sociología. ¿Cuántos alumnos llevan exactamente uno de tales cursos?
\begin{tasks}(5)
	\task{40}
	\task{44}
	\task{48}
	\task{52}
	\task{56}
\end{tasks}}
\end{itemize}
\end{document}
