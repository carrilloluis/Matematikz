\documentclass[11pt, a5paper]{article}

\usepackage[margin=.5in]{geometry}
\usepackage[utf8]{inputenc}
\usepackage[spanish]{babel}
\usepackage{amssymb, amsmath, amsbsy}
\usepackage{verbatim}
\usepackage{mathpazo}
\usepackage{tasks}
\usepackage{enumerate}
\usepackage[pdftex]{hyperref}
\hypersetup{pdftitle={Conceptos de aritmética básica (3)},pdfauthor={Luis Carrillo Gutiérrez}}

\renewcommand{\familydefault}{\sfdefault}

\begin{document}
\subsection*{Aritmética básica}
\begin{itemize}
\item{Dadas las siguientes proposiciones (en el conjunto de los números naturales $\mathbb{N}$) :
\begin{enumerate}[I.]
\item{El minuendo es igual al sustraendo más la diferencia.}
\item{El sustraendo es igual al minuendo menos la diferencia.}
\item{Si a la suma de dos números se le agrega su diferencia, el resultado es igual al duplo del minuendo.}
\item{Si a la suma de dos números se le resta su diferencia, el resultado es igual al duplo del sustraendo.}
\end{enumerate}
Indique si son falsas o verdaderas
\begin{tasks}(3)
\task{VFVF}
\task{VVVV}
\task{FVFF}
\task{VVFF}
\task{FVVV}
\end{tasks}
}
\item{La diferencia de dos números es $21$ y el número mayor excede a esta diferencia en $15$ ¿Cuáles son esos números?
\begin{tasks}(5)
\task{23; 2}
\task{32; 9}
\task{36; 15}
\task{39; 18}
\task{40; 20}
\end{tasks}
}
\end{itemize}
\end{document}
