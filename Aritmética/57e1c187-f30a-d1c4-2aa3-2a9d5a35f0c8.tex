
    11. ¿Cuántos divisores primos tiene:
       N =320?

    a) 1		b) 2		c) 3
    d) 4	e) 5


    12. Hallar (a + b) si MCM () = 525

       a) 4		b) 6		c) 8
       d) 10	            e) 12

    13. Calcular “a” si:
  = 100(2)

a) 1		b) 2		c) 3
d) 4		e) 5


    14. ¿Cuántos divisores compuestos tiene:
N = (5 x 72)2 ?

a) 10		b) 11		c) 12
d) 13		e) 14


    15. Hallar el valor de “n” si el MCM de A y B tiene 60 divisores.

A = 73 x 14
B = 7 x 2n x 3

a) 1		b) 2		c) 3
d) 4		e) 5

    16. Hallar el MCD de A y B:

A = 16 x 3
B = 8 x 15

a) 20		b) 16		c) 24
d) 30		e) 35

    17. Si:  N = 83 x 7 + 82 x 5 + 8x 4 + 20
Convertir N a base 8.

a) 7542(8)		        b) 5472(8)
c) 754(20)(8)                 d) 7564(8)
e) 8564(8)

    18. ¿Cuántos números compuestos hay en:
14, 25, 13, 16, 2, 1, 72?

a) 1		b) 2		c) 3
d) 4		e) 5

    19. Hallar el valor de  “n”  si el MCM de A y B, tiene 60 divisores.

A  =  2n + 1 x 34 x 7
B  =  22n x 35

a) 0		b) 1		c) 2
d) 3		e) 4

. Si:  N = 7 x 123 + 8 x 122 + 9 x 12 + 18
Convertir N a base 12.
a) 789(15)12	            d) 7996(12)
b) 7896(12)	             e) 789(10)(12)
c) 78(10)6(12)