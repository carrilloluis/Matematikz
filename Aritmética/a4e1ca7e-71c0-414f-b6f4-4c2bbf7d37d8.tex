\documentclass[11pt, a5paper]{article}

\usepackage[margin=.5in]{geometry}
\usepackage[utf8]{inputenc}
\usepackage[spanish]{babel}
\usepackage{amssymb, amsmath, amsbsy}
\usepackage{verbatim}
\usepackage{mathpazo}
\usepackage{tasks}
\usepackage{enumerate}
\usepackage[pdftex]{hyperref}
\hypersetup{pdftitle={Conceptos Básicos de Aritmética},pdfauthor={Luis Carrillo Gutiérrez}}

\renewcommand{\familydefault}{\sfdefault}

\begin{document}
\subsection*{Conceptos de Aritmética} % 2012
\begin{itemize}
\item{De las siguientes afirmaciones, determinar el valor de verdad i/o falsedad de las siguientes proposiciones.
\begin{enumerate}[I.]
\item{Todo número racional tiene su inverso multiplicativo.}
\item{Entre dos números racionales consecutivos existe siempre otro número racional.}
\item{El cero es un número racional.}
\item{El conjunto de los números racionales no es denso.}
\end{enumerate}
 \begin{tasks}(3)
\task{FVVF}
\task{FFFV}
\task{VVVF}
\task{VFVF}
\task{VFVV}
\end{tasks}
}
\item{De las siguientes proposiciones :
\begin{enumerate}[I.]
\item{Todo número entero es un número racional.}
\item{El sistema de los números racionales admite un elemento neutro multiplicativo.}
\item{El sistema de los números racionales es un conjunto denso.}
\item{El sistema de los números racionales, el cero no tiene inverso multiplicativo.}
\item{En el sistema de los números racionales es denso completo y ordenado.}
\item{La división está definida en el sistema de los números racionales.}
\end{enumerate}
¿Cuántas son proposiciones verdaderas?
\begin{tasks}(5)
\task{0}
\task{5}
\task{6}
\task{4}
\task{1}
\end{tasks}
}
\item{¿Cuántas de las siguientes proposiciones son falsas?
\begin{enumerate}[I.]
\item{Todo número natural es racional.}
\item{Todo número entero es racional.}
\item{El uno racional tiene la forma : $1 = a/a,\, \forall a \in \mathbb{Z}^{+}$}
\item{Todo número racional es real.}
\end{enumerate}
\begin{tasks}(5)
\task{0}
\task{1}
\task{2}
\task{4}
\task{5}
\end{tasks}
}
\item{¿Cuántas de las siguientes proposiciones son falsas?
\begin{enumerate}[I.]
\item{El número $11$ no es un número racional.}
\item{-12 no es un número racional.}
\item{15 es un número racional.}
\item{No existe fracción con denominador cero.}
\item{Entre dos números racionales consecutivos, siempre existe otro número racional.}
\end{enumerate}
\begin{tasks}(5)
\task{0}
\task{3}
\task{1}
\task{2}
\task{5}
\end{tasks}
}
\item{Dadas las siguientes proposiciones :
\begin{enumerate}[I.]
\item{Todo número fraccionario es un número racional y recíprocamente.}
\item{Toda fracción es un número fraccionario y todo número fraccionario es una fracción.}
\item{Toda operación realizada con un par de números racionales genera otro número racional.}
\end{enumerate}
Los respectivos valores de verdad son :
\begin{tasks}(3)
\task{VFV}
\task{FFF}
\task{VVV}
\task{VVF}
\task{FVF}
\end{tasks}
}
\item{Indicar el valor de verdad (V) o falsedad (F) de las siguientes proposiciones :
\begin{enumerate}
\item{En el sistema de los números racionales, la radicación está bien definida.}
\item{$\forall\, \dfrac{a}{b} \in \mathbb{Q},\,\exists \Big(\dfrac{a}{b}\Big)^{-1} \in \mathbb{Q}\,/\,\dfrac{a}{b} \times \Big(\dfrac{a}{b}\Big)^{-1} = 0$}
\item{El conjunto de los números racionales es denso.}
\item{$\forall\, \dfrac{a}{b} \in \mathbb{Q},\,\exists !1 \in \mathbb{Q}\,/\,\dfrac{a}{b} \times 1 = \dfrac{a}{b};b \neq 0$}
\end{enumerate}
\begin{tasks}(3)
\task{FVFV}
\task{FFVV}
\task{VFVV}
\task{VFFV}
\task{VVFV}
\end{tasks}
}
\end{itemize}
\end{document}