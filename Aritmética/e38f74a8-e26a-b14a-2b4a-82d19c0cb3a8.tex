\documentclass[11pt, twocolumn, landscape, a4paper]{article}

\usepackage[margin=1.2cm]{geometry}
\usepackage[utf8]{inputenc}
\usepackage[spanish]{babel}
\usepackage{amssymb, amsmath, amsbsy}
\usepackage{tasks}
\usepackage{enumerate}
\usepackage{graphicx}

\usepackage[pdftex]{hyperref}
\hypersetup{pdftitle={Aritmética - repaso},pdfauthor={Luis Carrillo Gutiérrez}}

\renewcommand{\familydefault}{\sfdefault}
\fontfamily{cmss}\selectfont
\pagestyle{empty}
 
\begin{document}
\subsection*{Aritmética}

\begin{itemize}

\item{Reducir la expresión : $f = \dfrac{2 \cdot (1{.}1 + \sqrt{0{.}21})^{3/2} (1{.}1 - \sqrt{0{.}21})^{3/2} }{3\sqrt{9}}$
\begin{tasks}(5)
\task{$1{.}21$}\task{$0{.}5$}\task{$\sqrt{1{.}2}$}\task{$\sqrt{0{.}5}$}\task{$0{.}21$}
\end{tasks}
}
\item{Determinar el tipo de expresión decimal que origina la siguiente fracción: \qquad$f = \dfrac{1}{2^{4} \cdot 5^{3} \cdot 3^{2} \cdot 11 \cdot 13}$
\begin{tasks}(1)
\task{inexacta periódica pura con 12 cifras en el período}
\task{inexacta periódica pura con 7 cifras en el período}
\task{inexacta periódica mixta con 4 cifras no períodicas}
\task{inexacta periódica mixta con 7 cifras periódicas}
\task{inexacta periódica mixta con 4 cifras no periódicas y 6 cifras en el período}
\end{tasks}
}
\item{¿Cuántas fracciones propias, irreductibles y homogéneas existen entre $\dfrac{17}{25}$ y $\dfrac{3}{4}$?
\begin{tasks}(5)
\task{2}\task{3}\task{4}\task{5}\task{7}
\end{tasks}
}
\item{La fracción $105/924$ genera un decimal periódico mixto, si ``$m$'' representa el número de cifras no periódicas y ``$p$'' el número de cifras del período. Hallar ``$m + p$''.
\begin{tasks}(5)
\task{2}\task{4}\task{5}\task{7}\task{12}
\end{tasks}
}
\item{¿Cuántas cifras no periódicas y periódicas tiene la expresión decimal de la fracción? $f = \dfrac{14}{2^{4} \times 13 \times 5^{6} \times 11}$
\begin{tasks}(5)
\task{6 y 6}\task{4 y 6}\task{4 y 5}\task{6 y 10}\task{6 y 12}
\end{tasks}
}
\end{itemize}
\end{document}