\documentclass[11pt, a5paper]{article}

\usepackage[margin=.5in]{geometry}
\usepackage[utf8]{inputenc}
\usepackage[spanish]{babel}
\usepackage{amssymb, amsmath, amsbsy}
\usepackage{mathpazo}
\usepackage{tasks}
\usepackage{enumerate}
\usepackage[pdftex]{hyperref}
\hypersetup{pdftitle={Aritmética en los Racionales},pdfauthor={Luis Carrillo Gutiérrez}}

\renewcommand{\familydefault}{\sfdefault}

\begin{document}
\subsection*{Aritmética básica}
\begin{itemize}
\item{Si a los dos términos de una fracción dada, se le resta $1$, el valor de la fracción resultante es $0{.}333\ldots$ y si a los dos términos de dicha fracción se aumenta $3$; el valor de la fracción obtenida es $0{.}5$. Entonces la suma de los términos de la fracción dada es :
\begin{tasks}(5)
\task{9}
\task{11}
\task{19}
\task{18}
\task{12}
\end{tasks}
}
\item{¿Cuántas fracciones menores a $9/10$ y mayores a $4/5$ existen, cuyos denominadores sean $60$ y que dichas fracciones sean irreductibles?
\begin{tasks}(5)
\task{3}
\task{2}
\task{6}
\task{4}
\task{5}
\end{tasks}
}
\item{¿Cuántas fracciones propias cuyos términos son enteros consecutivos son menores a $65/77$?
\begin{tasks}(5)
\task{4}
\task{5}
\task{2}
\task{3}
\task{6}
\end{tasks}
}
\item{El período de una fracción de denominador $11$ es de dos cifras que se diferencian en $5$ unidades. Hallar la suma de los términos de dicha fracción, si es la mayor posible.
\begin{tasks}(5)
\task{12}
\task{11}
\task{15}
\task{10}
\task{19}
\end{tasks}
}
\item{¿Cuántas fracciones propias, cuyos términos son consecutivos, son menores que $0{.}75$?
\begin{tasks}(5)
\task{1}
\task{2}
\task{3}
\task{4}
\task{5}
\end{tasks}
}
\end{itemize}
\end{document}