\documentclass[11pt, twocolumn, landscape, a4paper]{article}	

\usepackage[margin=1cm]{geometry}
\usepackage[utf8]{inputenc}
\usepackage[spanish]{babel}
\usepackage{amssymb, amsmath, amsbsy}
\usepackage{tasks}
\usepackage{enumerate}
\usepackage{graphicx}

\usepackage[pdftex]{hyperref}
\hypersetup{pdftitle={Aritmética - Varios},pdfauthor={Luis Carrillo Gutiérrez}}

\fontfamily{cmss}\selectfont
\renewcommand{\familydefault}{\sfdefault}
\pagestyle{empty}

\begin{document}
\subsection*{Aritmética - conjuntos}

\begin{itemize}
\item{Dados los conjuntos : 
\begin{eqnarray*}
\mathrm{A} &=& \{x \in \mathbb{Z}\,/\,2x \in \Big(\big< -2;\,3\big> * \big[-1;\,4\big>\Big)\} \\
\mathrm{B} &=& \{x \in \mathbb{Z}\,/\,(x \circ 2) \in \big<-1\,;\,8\big>\}
\end{eqnarray*}
Si \quad$\mathrm{P} * \mathrm{Q} = \mathrm{P} \cup (\mathrm{P}^{'} \cap \mathrm{Q})$\quad y \quad$a \circ b = a + b + ab$ \\
Donde $\mathrm{P}$ y $\mathrm{Q}$ son conjuntos; $a$ y $b$ números enteros. Señalar la alternativa que mejor corresponda:
\begin{tasks}(3)
\task{$\mathrm{A} = \mathrm{B}$}
\task{$\mathrm{A} \subset \mathrm{B}$}
\task{$\mathrm{B} \subset \mathrm{C}$}
\task{$\mathrm{A} - \mathrm{B} = \Phi$}
\task{$\mathrm{A}\,\Delta\,\mathrm{B}$}
% Todos son correctos
\end{tasks}
}
\item{Graficar los siguientes intervalos en la recta real y  efectuar las operaciones pertinentes: 
\begin{eqnarray*}
\mathrm{A} &=& \big[-2\,;\,10\big] \\
\mathrm{B} &=& \big[5\,;\,12\big] \\
\mathrm{C} &=& \big]9\,;\, \infty\big[ 
\end{eqnarray*}

\begin{itemize}
\item{$\mathrm{A} \cap \mathrm{B}$}
\item{$\mathrm{A} \cup \mathrm{B}$}
\item{$\mathrm{A} \cup \mathrm{B} \cup \mathrm{C}$}
\item{$\mathrm{A} \cap \mathrm{B} \cap \mathrm{C}$}
\item{$\mathrm{A} - \mathrm{B}$}
\item{$\mathrm{A} - (\mathrm{B} \cap \mathrm{C})$}
\item{$\mathrm{A}\,\Delta\,\mathrm{B}$}
\end{itemize}
}
\item{El subconjunto potencia de $\mathrm{M}$ tiene 28 subconjuntos binarios ¿Cuántos subconjuntos cuaternarios más que terciarios tiene el conjunto $\mathrm{M}$?
\begin{tasks}(5)
\task{7}\task{14}\task{21}\task{28}\task{72}
\end{tasks}
}
\item{Entre conjuntos se define $\mathrm{A} \perp \mathrm{B} = (\mathrm{A} - \mathrm{B})^{'}$. Determinar la verdad o falsedad de :
\begin{enumerate}[I.]
\item{$\mathrm{A}^{'} \perp \mathrm{A}^{'} = \mathrm{A}$}
\item{$(\mathrm{A} \perp \mathrm{A}) \perp (\mathrm{B} \perp \mathrm{B}) = \mathrm{A} \cup \mathrm{B}$}
\item{$(\mathrm{A} \perp \mathrm{B}) \perp (\mathrm{A} \perp \mathrm{B}) = \mathrm{A} - \mathrm{B}^{'}$}
\end{enumerate}
\begin{tasks}(5)
\task{FFF}\task{VVV}\task{VFV}\task{VVF}\task{FVV}
\end{tasks}
}
\item{Entre conjuntos se define : $\mathrm{A} * \mathrm{B} = \{x \in \mathrm{U}/x \in \mathrm{M}\,\wedge (x \in \mathrm{M}^{'} \vee x \in \mathrm{P}^{'})\}$. Escriba en operaciones usuales.
\begin{tasks}(3)
\task{$\mathrm{M}^{'} - \mathrm{P}$}
\task{$\mathrm{P}^{'} - \mathrm{M}$}
\task{$\mathrm{P} - \mathrm{M}^{'}$}
\task{$\mathrm{M} - \mathrm{P}$}
\task{$\mathrm{M}^{'} \cap\,\mathrm{M}$}
\end{tasks}
}
\end{itemize}
\end{document}
