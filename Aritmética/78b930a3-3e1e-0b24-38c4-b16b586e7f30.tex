\documentclass[11pt, a5paper]{article}

\usepackage[margin=.5in]{geometry}
\usepackage[utf8]{inputenc}
\usepackage[spanish]{babel}
\usepackage{amssymb, amsmath, amsbsy}
\usepackage{mathpazo}
\usepackage{tasks}
\usepackage{enumerate}
\usepackage[pdftex]{hyperref}
\hypersetup{pdftitle={Conceptos de aritmética},pdfauthor={Luis Carrillo Gutiérrez}}

\renewcommand{\familydefault}{\sfdefault}

\begin{document}

\subsection*{Aritmética básica (II)}
\begin{itemize}
\item{Indique los valores de verdad de las siguientes proposiciones :
\begin{enumerate}[I.]
\item{$(0{.}23223222322223\ldots + 0{.}01001000100001\ldots)$ es racional.}
\item{$4{.}0$ se aproxima a $\pi$ con un error menor que $4/5$.}
\item{Si $a$ y $b$ son irracionales, $a \neq b$, entonces $a{.}b$ es irracional.}
\end{enumerate}
\begin{tasks}(3)
\task{VVV}
\task{FFV}
\task{FVV}
\task{FVF}
\task{FFF}
\end{tasks}
}
\item{Dadas las proposiciones :
\begin{enumerate}[I.]
\item{Si $a$ y $b$ son irracionales, entonces $a + b$ es irracional.}
\item{Si $k \in \mathbb{Q}$ y $a$ es irracional $\implies a^k$ es irracional.}
\item{Si $k \in \mathbb{Q}$}, $k \neq 0$ y $a$ es irracional $\implies a{.}k$ es irracional.
\end{enumerate}
Los respectivos valores de verdad son :
\begin{tasks}(3)
\task{VVV}
\task{FFV}
\task{FVV}
\task{FVF}
\task{FFF}
\end{tasks}
}
\item{Dar el valor de verdad de las siguientes proposiciones :
\begin{enumerate}[I.]
\item{Todo número decimal se puede representar por una fracción.}
\item{Si $a < b \implies a^n < b^n$; $\forall\,a, b \in \mathbb{Q}; \forall\,n \in \mathbb{Z}$.}
\item{Si una fracción es propia, entonces es irreductible.}
\item{Si la suma de dos fracciones irreductibles da como resultado un número entero, entonces son fracciones homogéneas.}
\end{enumerate}
\begin{tasks}(3)
\task{FFVF}
\task{FFFF}
\task{FFVV}
\task{VFFV}
\task{VVFF}
\end{tasks}
}
\item{¿Cuáles de las siguientes proposiciones son verdaderas?
\begin{enumerate}[I.]
\item{$\forall\,a, b \in \mathbb{Z}$, ($a$ y $b$ diferentes) existe un número entero ``$x$'' tal que $a < x < b$.}
\item{Existe un número racional ``$b$'' diferente de cero, tal que $a{.b} = 1$, $\forall\,a$ y $b \in \mathbb{Z}$.}
\item{$a.b > 0 \implies (a < 0 \wedge b < 0) \vee (a > 0 \wedge b > 0)$.}
\item{Todos los subconjuntos del conjunto de los números racionales son densos.}
\end{enumerate}
\begin{tasks}(3)
\task{Solo I}
\task{Solo II}
\task{Solo II y III}
\task{Solo III}
\task{Solo IV}
\end{tasks}
}
\item{La fracción $\dfrac{2}{11}$ que característica(s) cumple :
\begin{tasks}
\task{Resulta un número decimal exacto de dos cifras periódicas.}
\task{Resulta un número decimal inexacto de tres cifras periódicas.}
\task{Resulta un número decimal de dos cifras no periódicas y una cifra periódica.}
\task{Resulta un número decimal inexacto de dos cifras periódicas.}
\task{Resulta un número decimal inexacto de una cifra periódica.}
\end{tasks}
}
\item{La fracción $\dfrac{3}{37}$ que característica(s) cumple :
\begin{tasks}
\task{Resulta un número decimal inexacto de dos cifras periódicas.}
\task{Resulta un número decimal exacto de dos cifras periódicas.}
\task{Resulta un número decimal exacto de tres cifras periódicas.}
\task{Resulta un número decimal inexacto de tres cifras periódicas.}
\task{Resulta un número entero sin cifras decimales.}
\end{tasks}
}
\end{itemize}
\end{document}