\documentclass[11pt, a5paper]{article}

\usepackage[margin=.75in]{geometry}
\usepackage[utf8]{inputenc}
\usepackage[spanish]{babel}
\usepackage{amssymb, amsmath, amsbsy}
\usepackage{verbatim}
\usepackage{mathpazo}
\usepackage{tasks}
\usepackage{enumerate}
\usepackage{graphicx} % \scalebox{1.4}{$$}
% \usepackage{tikz}
% \usetikzlibrary{angles,quotes}
\usepackage[pdftex]{hyperref}
\hypersetup{pdftitle={},pdfauthor={Luis Carrillo Gutiérrez}}

\renewcommand{\familydefault}{\sfdefault}

\begin{document}

\begin{itemize}
\item{Hallar ``$a + b$'', si: $\overline{ab}_{(9)} = 143_{(5)}$
\begin{tasks}(5)
\task{5}
\task{6}
\task{7}
\task{8}
\task{9}
\end{tasks}
\begin{eqnarray*}
\overline{ab}_{(9)} &=& a \cdot 9^{1} + b \cdot 9^{0} = (9a + b)_{(10)} \\
143_{(5)} &=& 1 \cdot 5^{2} + 4 \cdot 5^{1} + 3 \cdot 5^{0} \\
\empty &=& 25 + 20 + 3 = 48_{(10)} \\
\empty &=& (45 + 3)_{(10)} = (9 \cdot 5 + 3)_{(10)}
\end{eqnarray*}
$(9 \cdot 5 + 3)_{(10)} = (9a + b)_{(10)} \implies a = 5 \wedge b = 3$ \\ 
$\therefore\qquad a + b = ? \quad\to\quad 5 + 3 = 8$
}
\item{El producto de dos números es $1750$ y su MCM es $350$. \\ Hallar su MCD.
\begin{tasks}(5)
\task{2}
\task{3}
\task{5}
\task{7}
\task{11}
\end{tasks}
}
\item{ ¿ Cuántos divisores primos tiene : $N = 320$ ?
\begin{tasks}(5)
\task{1}
\task{2}
\task{3}
\task{4}
\task{5}
\end{tasks}
}
\item{La suma de $3$ números enteros consecutivos es $90$. Hallar el número intermedio.
\begin{tasks}(5)
\task{20}
\task{21}
\task{30}
\task{31}
\task{32} % N.A.
\end{tasks}
}
\item{Hallar el MCD de $A$ y $B$ si: \\
$A = 6 \times 14 \times 72$ \\ 
$B = 21 \times 11 \times 9$
\begin{tasks}(3)
\task{$3^{3} \times 2$}
\task{$3^{3} \times 7$}
\task{$2^{3} \times 3$}
\task{$2^{3} \times 3^{2}$}
\task{$11 \times 3^{2}$}
\end{tasks}
}
\begin{comment}
\item{Hallar “a + b”, si: = 143(5)
a) 5		b) 6		c) 7
d) 8		e) 9
}
\end{comment}
\item{Calcular el valor de $\beta$ si:
$N = 3 x 72 x 13$ tiene $30$ divisores.
a) 1
b) 2
c) 3
d) 4
e) 5
}
\begin{comment}
    8. Expresar  a base 8, si  es el mayor número posible.

a) 321(8)		b) 323(8)	c) 325(8)
d) 327(8)		e) 329(8)


    9. Hallar la cantidad de divisores primos de:
N = 21 x 14

a) 1		b) 2		c) 3
d) 4		e) 5


    10. Convertir:
1023(5) a base 25

a) 513(25)		b) 5(13) (25)	c) 6(13) (25)
d) 512(25)		e) 5(12) (25)


    11. ¿Cuántos divisores primos tiene:
       N =320?

    a) 1		b) 2		c) 3
    d) 4	e) 5


    12. Hallar (a + b) si MCM () = 525

       a) 4		b) 6		c) 8
       d) 10	            e) 12

    13. Calcular “a” si:
  = 100(2)

a) 1		b) 2		c) 3
d) 4		e) 5


    14. ¿Cuántos divisores compuestos tiene:
N = (5 x 72)2 ?

a) 10		b) 11		c) 12
d) 13		e) 14


    15. Hallar el valor de “n” si el MCM de A y B tiene 60 divisores.

A = 73 x 14
B = 7 x 2n x 3

a) 1		b) 2		c) 3
d) 4		e) 5

    16. Hallar el MCD de A y B:

A = 16 x 3
B = 8 x 15

a) 20		b) 16		c) 24
d) 30		e) 35

    17. Si:  N = 83 x 7 + 82 x 5 + 8x 4 + 20
Convertir N a base 8.

a) 7542(8)		        b) 5472(8)
c) 754(20)(8)                 d) 7564(8)
e) 8564(8)

    18. ¿Cuántos números compuestos hay en:
14, 25, 13, 16, 2, 1, 72?

a) 1		b) 2		c) 3
d) 4		e) 5

    19. Hallar el valor de  “n”  si el MCM de A y B, tiene 60 divisores.

A  =  2n + 1 x 34 x 7
B  =  22n x 35

a) 0		b) 1		c) 2
d) 3		e) 4

. Si:  N = 7 x 123 + 8 x 122 + 9 x 12 + 18
Convertir N a base 12.
a) 789(15)12	            d) 7996(12)
b) 7896(12)	             e) 789(10)(12)
c) 78(10)6(12)
\end{comment}
\end{itemize}
\end{document}