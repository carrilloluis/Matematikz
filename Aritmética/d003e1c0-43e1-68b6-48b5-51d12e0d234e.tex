\documentclass[11pt, a5paper]{article}

\usepackage[margin=.75in]{geometry}
\usepackage[utf8]{inputenc}
\usepackage[spanish]{babel}
\usepackage{amssymb, amsmath, amsbsy}
\usepackage{mathpazo}
\usepackage{tasks}
\usepackage{graphicx} % \scalebox{1.4}{$$}
\usepackage[pdftex]{hyperref}
\hypersetup{pdftitle={},pdfauthor={Luis Carrillo Gutiérrez}}

\renewcommand{\familydefault}{\sfdefault}

\begin{document}
\subsection*{Aritmética básica}
\begin{itemize}
\item{Hallar ``$a + b$'', si: $\overline{ab}_{(9)} = 143_{(5)}$
\begin{tasks}(5)
\task{5}
\task{6}
\task{7}
\task{8}
\task{9}
\end{tasks}
\begin{eqnarray*}
\overline{ab}_{(9)} &=& a \cdot 9^{1} + b \cdot 9^{0} = (9a + b)_{(10)} \\
143_{(5)} &=& 1 \cdot 5^{2} + 4 \cdot 5^{1} + 3 \cdot 5^{0} \\
\empty &=& 25 + 20 + 3 = 48_{(10)} \\
\empty &=& (45 + 3)_{(10)} = (9 \cdot 5 + 3)_{(10)}
\end{eqnarray*}
$(9 \cdot 5 + 3)_{(10)} = (9a + b)_{(10)} \implies a = 5 \wedge b = 3$ \\
$\therefore\qquad a + b = ? \quad\to\quad 5 + 3 = 8$
}
\item{ ¿ Cuántos divisores primos tiene : $N = 320$ ?
\begin{tasks}(5)
\task{1}
\task{2}
\task{3}
\task{4}
\task{5}
\end{tasks}
\begin{center}
	\begin{tabular}{l|r}
	320 & 2 \\
	160 & 2 \\
	80 & 2 \\
	40 & 2 \\
	20 & 2 \\
	10 & 2 \\
	5 & 5 \\
	1 & \empty
	\end{tabular}
\end{center}
$\therefore\quad 320 = 2^6 \cdot 5 \quad\implies\quad \omega(320) = 2$
}
\item{La suma de $3$ números enteros consecutivos es $90$. Hallar el número intermedio.
\begin{tasks}(5)
\task{20}
\task{21}
\task{30}
\task{31}
\task{32} % N.A.
\end{tasks}
\begin{eqnarray*}
(x - 1) + (x) + (x + 1) &=& 90 \\
x + x + x + 1 - 1 &=& 90 \\
3x &=& 90 \quad\implies\quad \fbox{x = 30} \\
\end{eqnarray*}
}
\item{Hallar el MCD de $A$ y $B$ si: \\
$A = 6 \times 14 \times 72$ \\
$B = 21 \times 11 \times 9$
\begin{tasks}(3)
\task{$3^{3} \times 2$}
\task{$3^{3} \times 7$}
\task{$2^{3} \times 3$}
\task{$2^{3} \times 3^{2}$}
\task{$3^{2} \times 11$}
\end{tasks}
\begin{eqnarray*}
A &=& 6 \times 14 \times 72 \quad\equiv\quad 2 \cdot 3 \times 2 \cdot 7 \times 2^3 \cdot 3^2 \\
A &=& 2^{\,5} \cdot 3^{\,3} \cdot 7 \\
B &=& 21 \times 11 \times 9 \quad\equiv\quad 7 \cdot 3 \times 11 \times 3^2 \\
B &=& 3^{\,3} \cdot 7 \cdot 11
\end{eqnarray*}
$\therefore\quad MCD(A, B) = 3^{\,3} \cdot 7 =  \quad{3^{\,3} \times 7}$
}
\item{Hallar la cantidad de divisores primos de: $N = 21 \times 14$
\begin{tasks}(5)
\task{1}
\task{2}
\task{3}
\task{4}
\task{5}
\end{tasks}
}
\item{Calcular el valor de $\beta$ si:
$N = 3^{\,\beta} \times 7^{\,2} \times 13$ tiene $30$ divisores.
\begin{tasks}(5)
\task{1}
\task{2}
\task{3}
\task{4}
\task{5}
\end{tasks}
}
\item{Expresar $\overline{abc}_{(6)}$ a base $8$, si $\overline{abc}_{(6)}$ es el mayor número posible.
\begin{tasks}(5)
\task{$321_{(8)}$}
\task{$323_{(8)}$}
\task{$325_{(8)}$}
\task{$327_{(8)}$}
\task{$329_{(8)}$}
\end{tasks}
}
\item{Convertir: $1023_{(5)}$ a base $25$
\begin{tasks}(3)
\task{$513_{(25)}$}
\task{$5(13)_{(25)}$}
\task{$6(13)_{(25)}$}
\task{$512_{(25)}$}
\task{$5(12)_{(25)}$}
\end{tasks}
}
\item{El producto de dos números es $1750$ y su MCM es $350$. \\ Hallar su MCD.
\begin{tasks}(5)
\task{2}
\task{3}
\task{5}
\task{7}
\task{11}
\end{tasks}
}
\item{Se quiere colocar 18 naranjas en cestas de forma que cada cesta contenga el mismo número de naranjas. ¿De cuántas maneras puede hacerlo?
}
\end{itemize}
\end{document}
