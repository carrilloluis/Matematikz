\documentclass[11pt, twocolumn, landscape, a4paper]{article}	

\usepackage[margin=1cm]{geometry}
\usepackage[utf8]{inputenc}
\usepackage[spanish]{babel}
\usepackage{amssymb, amsmath, amsbsy}
\usepackage{tasks}
\usepackage{enumerate}
\usepackage{graphicx}

\usepackage[pdftex]{hyperref}
\hypersetup{pdftitle={Seminario en Aritmética},pdfauthor={Luis Carrillo Gutiérrez}}

\fontfamily{cmss}\selectfont
\renewcommand{\familydefault}{\sfdefault}
\pagestyle{empty}

\begin{document}
\subsection*{Seminario en Aritmética}

\begin{itemize}
\item{Dado los conjuntos : $\mathrm{A} = \{1; 2; \{2; a\}; \{2; 1; b\} \}$. Señale cuál de las siguientes proposiciones es verdadera.
\begin{tasks}(3)
\task{$2 \in \{2; a\}$}
\task{$1 \in \{2; 1; b\}$}
\task{$2 \in \mathrm{A}$}
\task{$\{2; a\} \in \mathrm{A}$} % <---
\task{$\{2; a\} \in \{2; 1; b\}$}
\end{tasks}
}
\item{Dado el siguiente conjunto : $\mathrm{A} = \{1; 2; \{3\}; \{4; 5\}\}$, indique la relación falsa.
\begin{tasks}(3)
\task{$1 \in \mathrm{A}$}
\task{$\{3\} \in \mathrm{A}$}
\task{$\{4; 5\} \subset \mathrm{A}$} % <---
\task{$2 \in \mathrm{A}$}
\task{$\{\{3\}; \{4; 5\}\} \subset \mathrm{A}$}
\end{tasks}
}
\end{itemize}

% \begin{enumerate}
% \item{}
% \end{enumerate}  
\end{document}