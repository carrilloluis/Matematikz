\documentclass[11pt, a5paper]{article}

\usepackage[margin=.5in]{geometry}
\usepackage[utf8]{inputenc}
\usepackage[spanish]{babel}
\usepackage{amssymb, amsmath, amsbsy}
\usepackage{verbatim}
\usepackage{mathpazo}
\usepackage{tasks}
\usepackage{enumerate}
%\usepackage{graphicx} % \scalebox{1.4}{$$}
\usepackage[pdftex]{hyperref}
\hypersetup{pdftitle={Conceptos de aritmética básica},pdfauthor={Luis Carrillo Gutiérrez}}

\renewcommand{\familydefault}{\sfdefault}

\begin{document}
\subsection*{Aritmética básica}
\begin{itemize}
\item{La fracción $\dfrac{8}{11 \times 7}$ que característica(s) cumple :
\begin{tasks}
\task{Resulta un número decimal inexacto de cuatro cifras periódicas.}
\task{Resulta un número decimal inexacto de seis cifras periódicas.}
\task{Resulta un número decimal exacto de cinco cifras no periódicas.}
\task{Resulta un número decimal inexacto de seis cifras no periódicas.}
\task{Resulta un número decimal exacto de seis cifras no periódicas.}
\end{tasks}
}
\item{La fracción $\dfrac{11}{41 \times 13}$ que característica(s) cumple :
\begin{tasks}
\task{Resulta un número decimal inexacto de seis cifras periódicas.}
\task{Resulta un número decimal inexacto de treinta cifras periódicas.}
\task{Resulta un número decimal inexacto de venticuatro cifras no periódicas.}
\task{Resulta un número decimal exacto de treinta cifras periódicas.}
\task{Resulta un número decimal inexacto de doce cifras periódicas.}
\end{tasks}
}
\item{La fracción $\dfrac{1}{2^2 \times 5^3 \times 37 \times 41}$ que característica(s) cumple :
\begin{tasks}
\task{Resulta un número decimal exacto de quince cifras periódicas.}
\task{Resulta un número decimal inexacto de dos cifras no periódicas y quince cifras periódicas.}
\task{Resulta un número decimal inexacto de tres cifras no periódicas y quince cifras periódicas.}
\task{Resulta un número decimal exacto de tres cifras no periódicas y cuarentaicinco cifras periódicas.}
\task{Resulta un número decimal inexacto de cuatro cifras periódicas y quince cifras no periódicas.}
\end{tasks}
}  \newpage
\item{La fracción $\dfrac{3}{2^5 \times 5^4 \times 7 \times 11 \times 101}$ que característica(s) cumple :
\begin{tasks}
\task{Resulta un número decimal inexacto de cuatro cifras no periódicas y treinta cifras periódicas.}
\task{Resulta un número decimal inexacto de cinco cifras no periódicas y cuarenta cifras periódicas.}
\task{Resulta un número decimal inexacto de cinco cifras no periódicas y doce cifras periódicas.}
\task{Resulta un número decimal exacto de cinco cifras no periódicas y cuarenta cifras periódicas.}
\task{Resulta un número decimal exacto de cinco cifras no periódicas y doce cifras periódicas.}
\end{tasks}
}
\end{itemize}
\end{document}