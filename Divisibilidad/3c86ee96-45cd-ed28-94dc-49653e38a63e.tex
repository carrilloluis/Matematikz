\documentclass[11pt, twocolumn, landscape, a4paper]{article}	

\usepackage[margin=1.2cm]{geometry}
\usepackage[utf8]{inputenc}
\usepackage[spanish]{babel}
\usepackage{amssymb, amsmath, amsbsy}
\usepackage{tasks}
\usepackage{enumerate}
\usepackage{graphicx}

\usepackage[pdftex]{hyperref}
\hypersetup{pdftitle={Divisibilidad aritmética},pdfauthor={Luis Carrillo Gutiérrez}}

\fontfamily{cmss}\selectfont
\renewcommand{\familydefault}{\sfdefault}
\pagestyle{empty}

\begin{document}
\subsection*{Divisibilidad}
\begin{itemize}
\item{Indicar si es verdadero o falso en cada una de las proposiciones :
\begin{enumerate}[I.]
\item{Un número es múltiplo de 9 cuando la suma de las cifras da un múltiplo de 3.}
\item{Si un número termina en cero es múltiplo de 5.}
\item{Todo número que termina en 44 es múltiplo de 4.}
\end{enumerate}
\begin{tasks}(5)
\task{FFF}\task{VVV}\task{FVV}\task{FFV}\task{VFV}
\end{tasks}
}
\item{¿Cuál o cuáles de las siguientes proposiciones son verdaderas?
\begin{enumerate}[I.]
\item{Todo número entero no es múltiplo de la unidad.}
\item{El cero es múltiplo de cualquier número entero}
\item{Todo número entero es divisible por sí mismo.}
\item{El número $\overline{abcd}_{(n)}$ es $\dot{n} + d$}
\item{El número $24681 = 9 + 1$}
\end{enumerate}
\begin{tasks}(5)
\task{Todos}
\task{II, IV y V}
\task{I, II, III}
\task{sólo III}
\task{I, II y V}
\end{tasks}
}
\item{Siendo a, b y c tres números que son $\dot{3}$, entonces ¿cuál o cuáles de las siguientes expresiones son siempre $\dot{9}$?  
\begin{enumerate}[I.]
\item{$a + b + c$}
\item{$a^2 + b^2 + c^2$}
\item{$ab + bc + ac$}
\end{enumerate}
\begin{tasks}(5)
\task{I y II}
\task{II y III}
\task{Sólo I}
\task{Sólo II}
\task{Todas}
\end{tasks}
}
\end{itemize}
\end{document}