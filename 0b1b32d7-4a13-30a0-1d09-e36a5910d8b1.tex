\documentclass[10pt, twocolumn, landscape, a4paper]{article}

\usepackage[margin=1in]{geometry}
\usepackage[utf8]{inputenc}
\usepackage[spanish]{babel}
\usepackage{amssymb, amsmath, amsbsy}
\usepackage{verbatim}
\usepackage{mathpazo}
\usepackage{enumerate}
\usepackage{tasks}
\usepackage{tikz}
\usetikzlibrary{angles,quotes}
\usepackage[pdftex]{hyperref}
\hypersetup{ pdftitle = { Examen de Selección Docente / 2009 },pdfauthor = {Luis Carrillo Gutiérrez} }

\renewcommand{\familydefault}{\sfdefault}

\begin{document}
\subsection*{Examen de Selección Docente}
\begin{itemize}
\item{¿Cuántos palitos de fósforo se necesitan para formar la figura número $15$?
\begin{tasks}(5)
	\task{40}
	\task{46}
	\task{24}
	\task{52}
	\task{32}
\end{tasks}
}
\item{Observe las $3$ figuras siguientes :
Y determine ¿Cuál es el número de lados de la figura 200?
\begin{tasks}(5)
	\task{780}
	\task{801}
	\task{800}
	\task{799}
	\task{201}
\end{tasks}
}
\item{Dado el operador matemático definido por :
Halle :
\begin{tasks}(5)
	\task{5}
	\task{3}
	\task{1}
	\task{2}
	\task{4}
\end{tasks}}
\item{Se han comprado dos carros por S/. $40 000$. Uno de ellos costó $1/3$ del valor del otro. ¿Cuál es la diferencia de precios entre dichos carros?
\begin{tasks}(3)
	\task{S/. $12\,\,000$}
	\task{S/. $15\,\,000$}
	\task{S/. $20\,\,000$}
	\task{S/. $10\,\,000$}
	\task{S/. $18\,\,000$}
\end{tasks}
}
\item{Mis padres siempre anhelaron tener una docena de hijos; aunque no llegaron a dicho número. La tercera parte de mis hermanos son futbolistas y la quinta parte de mis hermanas son enfermeras. ¿Cuántos hijos somos, si mí nombre es Josefina?
\begin{tasks}(5)
	\task{8}
	\task{10}
	\task{9}
	\task{6}
	\task{11}
\end{tasks}
}
\item{En la figura, el lado de cada cuadradito es $1$ cm. Determina el área de la región sombreada.
\begin{tasks}(5)
	\task{$20\,\,cm^2$}
	\task{$21\,\,cm^2$}
	\task{$18\,\,cm^2$}
	\task{$22\,\,cm^2$}
	\task{$19\,\,cm^2$}
\end{tasks}
}
\item{Un comerciante vende un artículo en S/. $80$, ganando el $25$\% ¿Cuál fue el costo inicial?
\begin{tasks}(5)
	\task{S/. $78$}
	\task{S/. $64$}
	\task{S/. $18$}
	\task{S/. $20$}
	\task{S/. $35$}
\end{tasks}
}
\item{Halle el valor de ``$x$'' en la siguiente distribución :
\begin{tasks}(5)
	\task{76}
	\task{43}
	\task{72}
	\task{56}
	\task{42}
\end{tasks}}
\item{Un ómnibus que hace servicio de Puno a Juliaca, cobra S/. $2$ como pasaje único. En el trayecto, se observa que cada vez que baja un pasajero suben $2$. Si llega a Juliaca con $40$ pasajeros, y una recaudación de S/. $90$. ¿Cuántos pasajeros partieron de Puno?
\begin{tasks}(5)
	\task{15}
	\task{28}
	\task{35}
	\task{24}
	\task{18}
\end{tasks}
}
\item{Tres amigas contabilizan manzanas. Una de 4 en 4, otra de 5 en 5 y otra de 6 en 6, obteniéndose siempre el mismo número. ¿Cuántas manzanas hay?
\begin{tasks}(5)
	\task{80}
	\task{50}
	\task{60}
	\task{70}
	\task{90}
\end{tasks}
}
\item{En una escuela, hay un juego que consiste en trasladar los discos del primer eje al tercero. ¿Cuántos movimientos se deben realizar como mínimo, sabiendo que un disco grande no puede situarse sobre uno pequeño?
\begin{tasks}(5)
	\task{7}
	\task{5}
	\task{8}
	\task{4}
	\task{6}
\end{tasks}}

\item{El local principal de la \textit{Universidad Nacional del Altiplano} está al noreste del estadio \textit{Enrique Torres Belón}. La isla \textit{Estevés} está al sureste de la UNA y al este del estadio. ¿Cuál de las afirmaciones es correcta?
\begin{tasks}
	\task{La UNA está al noreste de la isla.}
	\task{El estadio está al oeste de la isla.}
	\task{El estadio está al sur de la UNA.}
	\task{El estadio está al este de la isla.}
	\task{La isla está al suroeste de la UNA.}
\end{tasks}}
\item{El gráfico circular siguiente muestra los resultados de na encuesta realizada a n alumnos sobre sus deportes favoritos. Si 35 prefieren básquet, Hallar el valor de n.
\begin{tasks}(5)
	\task{150}
	\task{120}
	\task{140}
	\task{100}
	\task{180}
\end{tasks}}
\item{Se arrojan dos dados. El resultado del lanzamiento del primer dado se multiplica por $8$ y se le suma el resultado del lanzamiento del segundo dado, obteniéndose $38$. La suma de los resultados de ambos lanzamientos es :
\begin{tasks}(5)
	\task{8}
	\task{6}
	\task{4}
	\task{10}
	\task{12}
\end{tasks}}
\item{Halle el valor de $x + y$ en la siguiente sucesión : $$2;\,5;\,8;\,20;\,32;\,80;\,x;\,y$$
\begin{tasks}(5)
	\task{352}
	\task{328}
	\task{356}
	\task{448}
	\task{446}
\end{tasks}}
\item{Con las frutas : manzana, plátano, papaya y piña. ¿Cuántos jugos de diferente sabor se podrán hacer?
\begin{tasks}(5)
	\task{4}
	\task{8}
	\task{16}
	\task{15}
	\task{14}
\end{tasks}}
\item{En la siguiente figura. ¿Cuál es el máximo número de cuadriláteros?
\begin{tasks}(5)
	\task{15}
	\task{12}
	\task{16}
	\task{14}
	\task{13}
\end{tasks}}
\item{La diablada del barrio ``Victoria'' estuvo constituída por $780$ personas; se sabe que por cada $6$ mujeres hablan $7$ varones. ¿Cuántas mujeres integraron la diablada?
\begin{tasks}(5)
	\task{130}
	\task{360}
	\task{180}
	\task{420}
	\task{210}
\end{tasks}}
\item{Pedro y Germán poseen la misma suma de dinero; pero Pedro tiene más dinero que María y está tiene más dinero que Manuel, Martha tiene más dinero que Manuel; pero menos que Pedro y menos que María. En conclusión :
\begin{tasks}
	\task{Manuel es más rico que Germán.}
	\task{Germán es más pobre que María.}
	\task{Pedro es más rico que Manuel.}
	\task{María es más pobre que Martha.}
	\task{Pedro tiene menos dinero que Manuel.}
\end{tasks}}
\item{La edad de Juan es el doble de la edad de María; y hace $15$ años, la edad de Juan era el triple de la edad de María ¿Cuántos años tiene María?
\begin{tasks}(5)
	\task{32}
	\task{30}
	\task{28}
	\task{24}
	\task{25}
\end{tasks}}
\item{Durante el gobierno del Partido Civil, la educación fue una preocupación especial para alcanzar los fines políticos y el desarrollo económico y social del país. Uno de los más destacados en este aspecto fue :
\begin{tasks}(2)
	\task{Mariano H. Cornejo.}
	\task{Manuel Vicente Villarán.}
	\task{Alejandro Octavio Deustúa.}
	\task{José Antonio Encinas.}
	\task{Manuel Z. Camacho.}
\end{tasks}}
\item{Las metas de la enseñanza de la historia, en las escuelas, están orientadas a capacitar a los niños para:
\begin{enumerate}[I. ]
\item{Establecer la secuencia.}
\item{Comprender los valores de nuestra sociedad.}
\item{Buscar explicaciones para el cambio.}
\item{Desarrollar la imaginación histórica.}
\item{Distinguir entre hechos históricos y su interpretación.}
\end{enumerate}

\begin{tasks}
	\task{II, V y III.}
	\task{III, IV y V.}
	\task{I, II y III.}
	\task{V, I y II.}
	\task{II, III y IV.}
\end{tasks}}
\end{itemize}
\end{document}