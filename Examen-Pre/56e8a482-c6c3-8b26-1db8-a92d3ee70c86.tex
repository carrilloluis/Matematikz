\documentclass[10pt, twocolumn, landscape, a4paper]{article}

\usepackage[margin=.65in]{geometry}
\usepackage[utf8]{inputenc}
\usepackage[spanish]{babel}
\usepackage{amssymb, amsmath, amsbsy}
\usepackage{verbatim}
\usepackage{mathpazo}
\usepackage{tasks}
\usepackage{enumerate}
\usepackage{graphicx}
\usepackage[pdftex]{hyperref}
\hypersetup{pdftitle = {Preguntas sobre Ciencias},pdfauthor = {Luis Carrillo Gutiérrez}}

\begin{document}

\subsection*{Física}
\begin{itemize}
\item{Una barra rígida que puede girar sobre un punto llamado \textit{eje} y su utilidad está en incrementar la efectividad de una fuerza aplicada que permite que dicha fuerza se aplique en lugar más conveniente es :

\begin{tasks}(2)
\task{El peso del cuerpo}
\task{Aceleración}
\task{Torque negativo}
\task{La palanca}
\task{Es un concepto ficticio}
\end{tasks}
}
\item{Se requiere  elevar un barril de $60\, Kg$ por un plano inclinado, cuya longitud es de $5\, m$  y cuya base es de $4\, m$ ¿Qué potencia se emplea si esta actúa paralelamente a la base?
\begin{tasks}(3)
\task{$46\, Kg$}
\task{$44\, Kg$}
\task{$45\, Kg$}
\task{$48\, Kg$}
\task{$43\, Kg$}
\end{tasks}
}
\item{Un alumno que pesa $120\, lbs$, se sostiene con ambos brazos de una barra horizontal ¿Qué fuerza ejerce cada brazo; si son paralelos y si cada uno forma con la vertical un ángulo de $30^{\circ}$ grados? % 40 lbs = 1 cm
\begin{tasks}(3)
\task{$60 - 68\,lbs$}
\task{$681 - 781\, lbs$}
\task{$601 - 781\, lbs$}
\task{$781 - 68\, lbs$}
\task{$701 - 781\, lbs$}
\end{tasks}
}
\end{itemize}

\subsection*{Ciencias naturales}
\begin{itemize}
\item{Marque lo correcto sobre la \textsc{Perénquima}
\begin{tasks}(2)
\task{No es el más abundante}
\task{No participa en la cicatrización}
\task{No cumple función de fotosíntesis}
\task{Cumple función de protección}
\task{Ninguno de los anteriores}
\end{tasks}
}
\item{Marque lo correcto sobre la raíz \textsc{hipogea}
\begin{tasks}(2)
\task{Es geotactica negativa}
\task{Es fotolnopica}
\task{Son de origen adventicial}
\task{El olluco es raíz hipogea}
\task{Todos son falsos}
\end{tasks}
}
\item{Marque lo correcto:

\begin{tasks}
\task{Todos los vertebrados no se originan de huevos.}
\task{Los ovovivíparos reciben nutrición de la sangre materna.}
\task{Los gemelos idénticos no son resultados de una reproducción sexual.}
\task{Todos los anteriores son verdaderos.}
\task{Todos los anteriores son falsos.}
\end{tasks}
}
\item{Marque la opción correcta :
\begin{tasks}
\task{Las proteínas globulares son proteínas conjugadas.}
\task{Las proteínas no pueden ser biocatalizadores.}
\task{Las proteínas simples tienen un grupo prostético.}
\task{Todos los anteriores son falsos.}
\task{Todos los anteriores son verdaderos.}
\end{tasks}
}
\item{Señale lo correcto : 
\begin{tasks}
\task{El sacro en su cara anterior presenta la cresta sacra.}
\task{El coxis se une por su vértice con el sacro.}
\task{El cuerpo del esternón se articula con la clavícula}
\task{La tuberosidad de las costillas se articula con el cuerpo vertebral.}
\task{Ninguna de las anteriores.}
\end{tasks}
}

\item{Los huesos del hombro se caracterizan por : 
\begin{tasks}
\task{El extremo externo de la clavícula se articula con la apófisis coracoides del omoplato.}
\task{La espina del omoplato se prolonga y termina en la apófisis coracoides.}
\task{La clavícula es un hueso corto cuyo borde superior se articula con el esternón.}
\task{La espina del omoplato divide en fosa suproespinosa y en fosa subcapilar.}
\task{Ninguna de las anteriores.}
\end{tasks}
}
\end{itemize}

\subsection*{Química - I}
\begin{itemize}
\item{Sobre la clasificación de los elementos, marque la relación \texttt{\textbf{no}} incorrecta : 
\begin{tasks}
\task{Newlands -- tríadas}
\task{Dobereiner -- cetavas}
\task{Moseley -- según el número atómico}
\task{Mendeliev -- según el peso atómico}
\task{Bohr -- mecánica no cuántica}
\end{tasks}
}
\item{Sobre el enlace químico, marque \texttt{V} o \texttt{F}.
\begin{enumerate}
\item{Enlace iónico es la coparticipación de pares de iones.}
\item{Enlace covalente predomina en química inórgánica.}
\item{Enlace covalente puro se produce entre átomos iguales.}
\item{Enlace covalente es la coparticipación de pares de electrones.}
\end{enumerate}

\begin{tasks}(3)
\task{VFFV}
\task{FVFV}
\task{FFVV}
\task{VVFF}
\task{FFFV}
\end{tasks}
}
\item{El nombre del compuesto $\;\mathsf{SO}_3\;$ es :
\begin{tasks}
\task{Trióxido de azufre (VI)}
\task{Óxido azúfrico}
\task{Anhídrido sulfúrico}
\task{Óxido de azufre (III)}
\task{Anhidrido de azufre (III)}
\end{tasks}
}
\end{itemize}

\subsection*{Trigonometría}
\begin{itemize}
\item{Una escalera de $30\, pies$ de longitud está apoyada en la pared de un edificio. ¿Cuál es la distancia del pie de la escalera al cimiento del edificio, si el ángulo que forma la escalera con la horizontal es de $60^{\circ}$?
\begin{tasks}(3)
\task{$30\, pies$}
\task{$60\, pies$}
\task{$15\, pies$}
\task{$20\, pies$}
\task{$25{.}85\, pies$}
\end{tasks}
}
\item{Un roca en la orilla de un río está a $132\, pies$ sobre el nivel del agua. Desde un punto opuesto de la roca, al otro lado del río se mide un ángulo de elevación de la cima de la roca de $14^{\circ} 16`$; Calcular el ancho aproximado del río. ($\cot 14^{\circ} 16` = 3{.}84$)
\begin{tasks}(3)
\task{$507\, pies$}
\task{$570\, pies$}
\task{$50{.}7\, pies$}
\task{$5{.}70\, pies$}
\task{$550\, pies$}
\end{tasks}
}
\item{Los dos lados iguales de un triángulo isósceles tienen cada uno $40\, cm$ de largo y los ángulos en la base miden $25^{\circ}$ cada uno. Hallar su área; si $\quad\sen 325^{\circ} = 0{.}42$ y $\quad\cos 25^{\circ} = 0{.}91$ 
\begin{tasks}(3)
\task{$720\,cm^2$}
\task{$620\,cm^2$}
\task{$672{.}60\,cm^2$}
\task{$611{.}52\,cm^2$}
\task{{\small Ninguna de\\ las anteriores}}
\end{tasks}
}
\end{itemize}

\subsection*{Geometría}
\begin{itemize}
\item{Los lados de un triángulo son $\overline{AB} = 8\, m$, $\overline{BC} = 6\,m$ y $\overline{AC} = 7\,m$ Se toma sobre $\overline{AB}$ los puntos $E$ y $M$; se trazan $\overline{EF}$, $\overline{MN}$ paralelas a $\overline{AC}$. Si $\overline{MN} - \overline{EF} = 2\,m$. Calcular la longitud $\overline{EM}$.
\begin{tasks}(5)
\task{$\dfrac{7}{3}$}
\task{$\dfrac{5}{3}$}
\task{$\dfrac{11}{3}$}
\task{$\dfrac{8}{3}$}
\task{$\dfrac{1}{3}$}
\end{tasks}
}
\item{Los lados $\overline{AB}$ y $\overline{BC}$ de un rectángulo miden $4\, cm$ y $12\, cm$ respectivamente. Sobre $\overline{AD}$ se toma $\overline{AF} = \overline{FG} = \overline{GD} = 4\, m$, $\overline{BF}$ y $\overline{BG}$ cortan a la diagonal $\overline{AC}$ en $E$ y $P$. Calcular $\overline{EP}$.
\begin{tasks}(3)
\task{$\dfrac{8}{5}\sqrt{10}$}
\task{$\dfrac{3}{5}\sqrt{10}$}
\task{$\dfrac{7}{5}\sqrt{10}$}
\task{$\dfrac{11\cdot \sqrt{10}}{5}$}
\task{{\small Ninguna de\\ los anteriores}}
\end{tasks}
}
\end{itemize}

\subsection*{Habilidad operativa}
\begin{itemize}
\item{Hallar el valor de \scalebox{1.2}{$\;x\;\,$} si : $\dfrac1{x} = [2 - 4{.}5^{-1}]$
\begin{tasks}(3)
\task{$\dfrac{16}{9}$}
\task{$\dfrac{20}{9}$}
\task{$\dfrac{17}{9}$}
\task{$1$}
\task{{\small Ninguna de\\ los anteriores}}
\end{tasks}}
\item{Hallar \scalebox{1.2}{$\;x\;$} si : $\quad - \sqrt{x} - \sqrt{x + 7} = -7$
\begin{tasks}(3)
\task{$8$}
\task{$11$}
\task{$23$}
\task{$3593$}
\task{$0$}
\end{tasks}
}
\item{El resultado de : $\quad 10 - 8 \times 3 \div 2 + 3 \times 9 - 20 \div 4 - 20$
\begin{tasks}(5)
\task{$0$}
\task{$-\dfrac{19}{4}$}
\task{$5$}
\task{$2$}
\task{$1$}
\end{tasks}
}
\item{Resolver $(1 -i)^4$ donde : $\quad i = \sqrt{-1}$
\begin{tasks}(5)
\task{$-4$}
\task{$0$}
\task{$4$}
\task{$\sqrt{4\;i\,}$}
\task{$1-i$}
\end{tasks}
}
% $1200{.}00$
\item{Un comerciante compró algunas cabras por \,\textsf{S/.} $1200$ y las vendió por \,\textsf{S/.} $1500$ ganando \,\textsf{S/.} $50$ en cada cabra ¿Cuántas cabras había originalmente?
\begin{tasks}(3)
\task{6}
\task{54}
\task{8}
\task{15}
\task{12}
\end{tasks}
}
\end{itemize}

\end{document}
