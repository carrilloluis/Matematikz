\documentclass[11pt, twocolumn, landscape, a4paper]{article}

\usepackage[margin=.65in]{geometry}
\usepackage[utf8]{inputenc}
\usepackage[spanish]{babel}
\usepackage{amssymb, amsmath, amsbsy}
\usepackage{verbatim}
\usepackage{mathpazo}
\usepackage{tasks}
\usepackage{enumerate}
\usepackage{graphicx} % \scalebox{1.4}{$$}
% \usepackage{tikz}
% \usetikzlibrary{angles,quotes}
\usepackage[pdftex]{hyperref}
\hypersetup{ pdftitle = {   },pdfauthor = {Luis Carrillo Gutiérrez} }

\begin{document}

\subsection*{Física}
\begin{itemize}
\item{Una barra rígida que puede girar sobre un punto llamado \textit{eje} y su utilidad está en incrementar la efectividad de una fuerza aplicada que permite que dicha fuerza se aplique en lugar más conveniente es :
}
\item{Se requiere  elevar un  barril de 60Kg por un plano inclinado, cuya longitud es de 5m y cuya base es de 4m ¿Qué potencia se emplea di esta actúa paralelamente a la base?}
\item{Un alumno que pesa 120 lbs, se sostiene con ambos brazos de una barra horizontal ¿Qué fuerza ejerce cada brazo ; si son paralelos y si cada una forma con la vertical un ángulo de 30 grados?}
\end{itemize}

\end{document}