\documentclass[11pt, twocolumn, landscape, a4paper]{article}

\usepackage[margin=.8in]{geometry}
\usepackage[utf8]{inputenc}
\usepackage[spanish]{babel}
\usepackage{amssymb, amsmath, amsbsy}
\usepackage{verbatim}
\usepackage{mathpazo}
\usepackage{tasks}
\usepackage{enumerate}
\usepackage{graphicx} % \scalebox{1.4}{$$}
% \usepackage{tikz}
% \usetikzlibrary{angles,quotes}
\usepackage[pdftex]{hyperref}
\hypersetup{ pdftitle = {Razomiento Matemático / 1989},pdfauthor = {Luis Carrillo Gutiérrez} }

\renewcommand{\familydefault}{\sfdefault}

\begin{document}
\subsection*{Razomiento matemático}
\begin{itemize}
\item{En un colegio se sabe que los dos tercios ($2/3$) de los estudiantes estudian inglés, la quinta parte ($1/5$) castellano y la sexta parte ($1/6$) ambos idiomas. ¿Qué fracción de los estudiantes no estudian ninguno de los idiomas citados?
\begin{tasks}(3)
\task{$30/10$}
\task{$10/30$}
\task{$3/10$}
\task{$20/10$}
\task{$10/15$}
\end{tasks}
}

\item{Sobre una recta se dan los puntos consecutivos $P$, $Q$, $R$ y $S$; si $R$ es el punto medio de $\overline{PS}$. ¿Cuál será el valor del segmento $\overline{QS}$?
\begin{tasks}(2)
\task{$\overline{QS} - \overline{PQ} / 2$}
\task{$2\overline{PQ}$}
\task{$1/2$}
\task{$10$}
\task{Ninguna de las anteriores}
\end{tasks}
}

\item{En una recta se toman los puntos consecutivos $A$, $B$, $C$ y $D$; de tal manera que : $\overline{AB} = 2\, \text{metros}$, $\overline{CD} = 3\, \text{metros}$ y
$\dfrac1{AB} + \dfrac1{AD} = \dfrac{2}{AC}$\, Calcular la longitud en metros de $\overline{BC}$. 
\begin{tasks}(3)
\task{1}
\task{2}
\task{3}
\task{4}
\task{Ninguna de\\ las anteriores}
\end{tasks}
}

\item{Hallar el número de puntos de intersección de 24 rectas secantes.
\begin{tasks}(3)
\task{275}
\task{276}
\task{277}
\task{278}
\task{279}
\end{tasks}
}

\item{Si $48$ miembros de una asamblea votan a favor de una moción y $12$ votan en contra. ¿Qué fracción votó a favor de la moción?
\begin{tasks}(3)
\task{$3/5$}
\task{$12/25$}
\task{$1/4$}
\task{$8/25$}
\task{$4/5$}
\end{tasks}
}

\item{¿Cuál es la octava parte de un octavo?\begin{tasks}(5)
\task{$1$}
\task{$1/16$}
\task{$1/4$}
\task{$1/64$}
\task{$64$}
\end{tasks}
}

\item{¿A cuántos tercios es igual $27$?
\begin{tasks}(5)
\task{$81$}
\task{$81/3$}
\task{$9$}
\task{$27/3$}
\task{$323$}
\end{tasks}
}

\item{Si la longitud de la diagonal de un cuadrado es $\overline{AB}$, entonces ¿Cuál es el área del cuadrado?
\begin{tasks}(5)
\task{$\overline{AB}$}
\task{$\dfrac1{2}\overline{AB}$}
\task{$\sqrt{2}\cdot\overline{AB}$}
\task{$\dfrac1{4}\overline{AB}$}
\task{$\overline{AB}^{\,2}$}
\end{tasks}
}

\item{Perdí ($1/5$) de ($1/2$) de S/. $280$  ¿Cuánto me quedó?
\begin{tasks}(5)
\task{$256$}
\task{$273$}
\task{$150$}
\task{$272$}
\task{$170$}
\end{tasks}
}

\item{¿Cuál es el número que añadido a $9$ dá dos veces el tercio ($1/3$) de $24$?
\begin{tasks}(5)
\task{$8$}
\task{$11$}
\task{$9$}
\task{$5$}
\task{$7$}
\end{tasks}
}
\item{Un alumno multiplico un número por $20$ en lugar de dividirlo entre $20$. La respuesta que obtuvo fue de $600$. ¿Cuál habría sido la respuesta correcta?
\begin{tasks}(5)
\task{$3$}
\task{$3000$}
\task{$0.3$}
\task{$1.5$}
\task{$15.6$}
\end{tasks}
}

\item{En un negocio José pierde (\scalebox{1.2}{$x/n$}) parte del capital, si aún le quedan \scalebox{1.2}{$z$} soles. ¿Cuánto tenía José al empezar el negocio?
\begin{tasks}(3)
\task{$\dfrac{nz}{n - x}$}
\task{$\dfrac{nx}{n - z}$}
\task{$\dfrac{nz}{x - n}$}
\task{$\dfrac{nzx}{nx}$}
\task{Ninguna de\\ las anteriores}
\end{tasks}}

\item{¿Qué número elevado a los exponentes $3$ y $-3$, dá como resultado una misma cantidad?
\begin{tasks}(5)
\task{$1$}
\task{$0.1$}
\task{$1.1$}
\task{$3$}
\task{($1/3$)}
\end{tasks}}

\item{Hallar : \qquad\scalebox{1.4}{$\sqrt[5]{\sqrt[3]{\sqrt{\sqrt[4]{x^{\,240}}}}}$}
\begin{tasks}(5)
\task{$x$}
\task{$-x$}
\task{$x^2$}
\task{$\sqrt{x}$}
\task{$x^5$}
\end{tasks}}

\item{¿Cuál es el número que al doble de su suma con $7$ es igual a tres veces su diferencia con $10$?
\begin{tasks}(5)
\task{$16$}
\task{$17$}
\task{$24$}
\task{$44$}
\task{$64$}
\end{tasks}}

\item{Hallar el valor de $x$ en :
\begin{tasks}(2)
\task{$10$}
\task{$24$}
\end{tasks}}
\end{itemize}
\end{document}