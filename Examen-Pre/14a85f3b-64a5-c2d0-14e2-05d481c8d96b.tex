\documentclass[11pt, twocolumn, landscape, a4paper]{article}

\usepackage[margin=.65in]{geometry}
\usepackage[utf8]{inputenc}
\usepackage[spanish]{babel}
\usepackage{amssymb, amsmath, amsbsy}
\usepackage{verbatim}
\usepackage{mathpazo}
\usepackage{tasks}
\usepackage{enumerate}
\usepackage{graphicx} % \scalebox{1.4}{$$}
% \usepackage{tikz}
% \usetikzlibrary{angles,quotes}
\usepackage[pdftex]{hyperref}
\hypersetup{ pdftitle = {   },pdfauthor = {Luis Carrillo Gutiérrez} }

\renewcommand{\familydefault}{\sfdefault}

\begin{document}

\subsection*{Lenguaje}
\begin{itemize}

\item{Señale la oración cuyo predicado tenga \textsc{modificación circunstancial} :
\begin{tasks}(2)
\task{no quiero escribir esa carta}
\task{el fútbol es un deporte masivo}
\task{la conquista aplastó la independencia de la sociedad andina}
\task{el equipo de tintaya espera su clasificación}
\task{arroz con leche es el postre de la casa}
\end{tasks}
}

\item{La palabra subrrayada funciona como : ``Ella vino temprano y \underline{les} dijo muchos concejos''.
\begin{tasks}(2)
\task{objeto directo}
\task{objeto indirecto}
\task{circunstancial de causa}
\task{agente}
\task{circunstancial de tema}
\end{tasks}
}

\item{``\texttt{\,Llueve\,}'' es :
\begin{tasks}(2)
\task{Un verbo personal}
\task{Una oración bimembre}
\task{Una oración unimembre}
\task{Una frase }
\task{Un verbo regular}
\end{tasks}
}
\end{itemize}

\subsection*{Literatura}
\begin{itemize}
\item{Una de las obras representativas de \textsc{Luis Alberto Sanchez} es :
\begin{tasks}(2)
\task{Los heraldos negros}
\task{Redoble por rancas}
\task{Ríos profundos}
\task{Los perros hambrientos}
\task{Aladino}
\end{tasks}
}

\item{\textsc{José Galvez Barrenechea} obtuvo resonante éxito popular estudiantil con su obra :
\begin{tasks}(2)
\task{Canción a la patria}
\task{Canción al amor}
\task{Canción a la juventud}
\task{Canción a la vida}
\task{Canción a la primavera}
\end{tasks}
}

\item{\textsc{Pablo Neruda} recibió como reconocimiento a su labor literaria :
\begin{tasks}(2)
\task{Premio nobel de literatura}
\task{El sol de los andes}
\task{Premio príncipe de asturias}
\task{Premio literario planeta}
\task{La orden del sol}
\end{tasks}
}
\end{itemize}

\subsection*{Historia peruana}
\begin{itemize}
\item{Señale la relación correcta :
\begin{tasks}(2)
\task{Sur/Este -- Collasuyo}
\task{Sur/Este -- Chinchaysuyo}
\task{Sur/Este -- Antisuyo}
\task{Sur/Este -- Contisuyo}
\task{Ninguno de los anteriores}
\end{tasks}
}

\item{Formado por los cuatro jefes de los suyos y asesoraban al Inca :
\begin{tasks}(2)
\task{Tojrikus}
\task{Camachico}
\task{Huno Camayoc}
\task{Consejo Imperial}
\task{Panaca}
\end{tasks}
}

\item{Se dedicaban a la agricultura y eran la mayoría del tawantinsuyo.}
\end{itemize}

\subsection*{Geografía (peruana)}
\begin{itemize}
\item{La Cordillera de los Andes se ha formado a través de :
\begin{tasks}(2)
\task{movimientos sísmicos}
\task{movimientos orogénicos}
\task{movimientos epirogénicos}
\task{movimientos de equilibrio}
\task{ninguno de los anteriores}
\end{tasks}
}

\item{Uno de los  principales picos de la cordillera de los Andes, en la sección occidental del Norte es :
\begin{tasks}(3)
\task{Omate}
\task{Ampato}
\task{Salkantay}
\task{Ananea}
\task{Yerupajá}
\end{tasks}
}
\item{Uno de los valles interandinos de los Andes del Norte es :
\begin{tasks}(2)
\task{Pampas}
\task{Huarpa}
\task{Callejón de Huaylas}
\task{Mantaro}
\task{Ninguna de las anteriores}
\end{tasks}
}
\end{itemize}

\subsection*{Economía política}
\begin{itemize}

\item{La crísis económica es :
\begin{tasks}
\task{El desequilibrio del estado}
\task{El punto más bajo de la depresión económico}
\task{Perturbaciones de las actividades financieras del país}
\task{Todas las anteriores}
\task{Ninguna de las anteriores}
\end{tasks}
}

\item{Los elementos fundamentales del \textsc{proceso productivo} son :
\begin{tasks}
\task{Naturaleza, trabajo, capital empresa, estado}
\task{Tierra, trabajo, capital, empresa, estado}
\task{Producción, circulación, consumo, distribución, inversión}
\task{}
\task{}
\end{tasks}
}

\end{itemize}

\end{document}