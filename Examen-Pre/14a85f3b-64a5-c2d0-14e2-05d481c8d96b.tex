\documentclass[10pt, twocolumn, landscape, a4paper]{article}

\usepackage[margin=.65in]{geometry}
\usepackage[utf8]{inputenc}
\usepackage[spanish]{babel}
\usepackage{amssymb, amsmath, amsbsy}
\usepackage{verbatim}
% \usepackage{mathpazo}
\usepackage{tasks}
\usepackage{enumerate}
\usepackage{graphicx} % \scalebox{1.4}{$$}
% \usepackage{tikz}
% \usetikzlibrary{angles,quotes}
\usepackage[pdftex]{hyperref}
\hypersetup{ pdftitle = {   },pdfauthor = {Luis Carrillo Gutiérrez} }

% \renewcommand{\familydefault}{\sfdefault}

\begin{document}

\subsection*{Lenguaje}
\begin{itemize}

\item{Señale la oración cuyo predicado tenga \textsc{modificación circunstancial} :
\begin{tasks}(2)
\task{no quiero escribir esa carta}
\task{el fútbol es un deporte masivo}
\task{la conquista aplastó la independencia de la sociedad andina}
\task{el equipo de tintaya espera su clasificación}
\task{arroz con leche es el postre de la casa}
\end{tasks}
}

\item{La palabra subrrayada funciona como : ``Ella vino temprano y \underline{les} dijo muchos concejos''.
\begin{tasks}(2)
\task{objeto directo}
\task{objeto indirecto}
\task{circunstancial de causa}
\task{agente}
\task{circunstancial de tema}
\end{tasks}
}

\item{``\texttt{\,Llueve\,}'' es :
\begin{tasks}(2)
\task{Un verbo personal}
\task{Una oración bimembre}
\task{Una oración unimembre}
\task{Una frase }
\task{Un verbo regular}
\end{tasks}
}
\end{itemize}

\subsection*{Literatura}
\begin{itemize}
\item{Una de las obras representativas de \textsc{Luis Alberto Sanchez} es :
\begin{tasks}(2)
\task{Los heraldos negros}
\task{Redoble por rancas}
\task{Ríos profundos}
\task{Los perros hambrientos}
\task{Aladino}
\end{tasks}
}

\item{\textsc{José Galvez Barrenechea} obtuvo resonante éxito popular estudiantil con su obra :
\begin{tasks}(2)
\task{Canción a la patria}
\task{Canción al amor}
\task{Canción a la juventud}
\task{Canción a la vida}
\task{Canción a la primavera}
\end{tasks}
}

\item{\textsc{Pablo Neruda} recibió como reconocimiento a su labor literaria :
\begin{tasks}(2)
\task{Premio nobel de literatura}
\task{El sol de los andes}
\task{Premio príncipe de asturias}
\task{Premio literario planeta}
\task{La orden del sol}
\end{tasks}
}
\end{itemize}

\subsection*{Historia (peruana)}
\begin{itemize}
\item{Señale la relación correcta :
\begin{tasks}(2)
\task{Sur/Este -- Collasuyo}
\task{Sur/Este -- Chinchaysuyo}
\task{Sur/Este -- Antisuyo}
\task{Sur/Este -- Contisuyo}
\task{Ninguno de los anteriores}
\end{tasks}
}

\item{Formado por los cuatro jefes de los suyos y asesoraban al Inca :
\begin{tasks}(2)
\task{Tojrikus}
\task{Camachico}
\task{Huno Camayoc}
\task{Consejo Imperial}
\task{Panaca}
\end{tasks}
}

\item{Se dedicaban a la agricultura y eran la mayoría del tawantinsuyo.}
\end{itemize}

\subsection*{Geografía (peruana)}
\begin{itemize}
\item{La Cordillera de los Andes se ha formado a través de :
\begin{tasks}(2)
\task{movimientos sísmicos}
\task{movimientos orogénicos}
\task{movimientos epirogénicos}
\task{movimientos de equilibrio}
\task{ninguno de los anteriores}
\end{tasks}
}

\item{Uno de los  principales picos de la cordillera de los Andes, en la sección occidental del Norte es :
\begin{tasks}(3)
\task{Omate}
\task{Ampato}
\task{Salkantay}
\task{Ananea}
\task{Yerupajá}
\end{tasks}
}
\item{Uno de los valles interandinos de los Andes del Norte es :
\begin{tasks}(2)
\task{Pampas}
\task{Huarpa}
\task{Callejón de Huaylas}
\task{Mantaro}
\task{Ninguna de las anteriores}
\end{tasks}
}
\end{itemize}

\subsection*{Economía política}
\begin{itemize}

\item{La crísis económica es :
\begin{tasks}
\task{El desequilibrio del estado}
\task{El punto más bajo de la depresión económico}
\task{Perturbaciones de las actividades financieras del país}
\task{Todas las anteriores}
\task{Ninguna de las anteriores}
\end{tasks}
}

\item{Los elementos fundamentales del \textsc{proceso productivo} son :
\begin{tasks}
\task{Naturaleza, trabajo, capital empresa, estado}
\task{Tierra, trabajo, capital, empresa, estado}
\task{Producción, circulación, consumo, distribución, inversión}
\task{Medios de producción y de trabajo}
\task{Producción, intercambio y capital}
\end{tasks}
}
\item{El rol directo del trabajo es la explotación, transformación de recursos :
\begin{tasks}
\task{La creación de máquinas y de equipos.}
\task{La disminución de la inflación.}
\task{La planificación de la economía nacional.}
\task{La elevación de la utilidad y del valor.}
\task{La amortización de la deuda externa.}
\end{tasks}
}
\end{itemize}

\subsection*{Educación cívica}
\begin{itemize}
\item{Los países participantes del \textsc{Grupo Andino}, llamados también \textsc{Pacto Andino} son :
\begin{tasks}
\task{Argentina, Bolivia, Chile, Ecuador, Perú.}
\task{Colombia, Panamá, Ecuador, Perú, Chile.}
\task{Bolivia, Colombia, Ecuador, Perú, Venezuela.}
\task{Argentina, Brasil, Perú, Venezuela, Colombia.}
\task{Bolivia, Chile, Ecuador, Colombia, Perú, Venezuela.}
\end{tasks}
}
\item{La \textsc{acción de amparo} tiene por finalidad :
\begin{tasks}
\task{Proteger el funcionamiento del tribunal de garantías constitucionales.}
\task{Sólo la libertad individual.}
\task{Sólo la libertad de expresión.}
\task{Asegurar el funcionamiento del congreso nacional.}
\task{Proteger las libertades reconocidas por la constitución a excepción de la individual.}
\end{tasks}
}
\item{Los \textsc{derechos humanos} fueron originariamente reconocidos por
\begin{tasks}
\task{En la constitución norteamericana de 1776.}
\task{En la declaración americana de los derechos y deberes del hombre, Bógota - Colombia en 1948.}
\task{La declaración universal de los derechos humanos aprobada por la ONU el 5 de febrero de 1951.}
\task{La declaración de los derechos del hombre, del ciudadano por la Asamblea Nacional Francesa en 1789.}
\task{En la declaración de los derechos humanos en San José de Costa Rica en 1948.}
\end{tasks}
}
\end{itemize}

\subsection*{Filosofía}
\begin{itemize}
\item{``Dar a cada uno, según sus necesidades'', es un enunciado ético que ejemplifica :
\begin{tasks}(2)
\task{La justicia - conmutativa}
\task{La justicia - distributiva}
\task{Una verdad de la filosofía - política}
\task{La justicia divina}
\task{Ninguna de\\ las anteriores}
\end{tasks}
}
\item{Es condición fundamental del deber que la persona sea : 
\begin{tasks}(3)
\task{Mayor de edad}
\task{Libre}
\task{Consciente}
\task{Educada}
\task{Casada}
\end{tasks}
}
\item{El principio de contradicción se expresa en lenguaje lógico 
\begin{tasks}(5)
\task{$p \wedge p$}
\task{$p \wedge \neg p$}
\task{$p \to \neg p$}
\task{$p \vee \neg p$}
\task{$\neg p$}
\end{tasks}
}
\end{itemize}

\subsection*{Psicología}
\begin{itemize}
\item{¿Cuál de las siguientes características pertenece a los fenómenos psíquicos?
\begin{tasks}(3)
\task{Subjetivos}
\task{Intransferibles}
\task{Temporales}
\task{Espaciales}
\task{No ser perceptibles\\ por los sentidos}
\end{tasks}
}
\item{¿Cómo se denomina a los conocimientos que adquirimos de manera natural y espontánea?
\begin{tasks}(3)
\task{Herencia}
\task{Aprendizaje}
\task{Socialización}
\task{Maduración}
\task{Todas las \\opciones anteriores}
\end{tasks}
}
\item{¿Cuál de las siguientes características no pertenece al acto de conocer?
\begin{tasks}(3)
\task{Subjetivo}
\task{Temporal}
\task{Transferible}
\task{Teórico}
\task{Psíquico}
\end{tasks}
}
\item{La conducta sexual permite descubrir en el hombre fundamentalmente su dimensión :
\begin{tasks}(3)
\task{Espiritual}
\task{Histórica}
\task{Social}
\task{Biológica}
\task{Física}
\end{tasks}
}
\item{Una causa frecuente de los conflictos sociales en nuestro medio es o son :
\begin{tasks}
\task{El segregacionismo racial.}
\task{El analfabetismo.}
\task{Los conflictos religiosos.}
\task{La desigualdad en la distribución de la riqueza social.}
\task{Los actos de gobierno.}
\end{tasks}
}
\end{itemize}
\end{document}
