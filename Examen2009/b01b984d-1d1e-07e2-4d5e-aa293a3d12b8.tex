\documentclass[10pt, twocolumn, landscape, a4paper]{article}

\usepackage[margin=.8in]{geometry}
\usepackage[utf8]{inputenc}
\usepackage[spanish]{babel}
\usepackage{amssymb, amsmath, amsbsy}
\usepackage{verbatim}
\usepackage{mathpazo}
\usepackage{enumerate}
\usepackage{tasks}
\usepackage{tikz}
\usetikzlibrary{angles,quotes}
\usepackage[pdftex]{hyperref}
\hypersetup{ pdftitle = { Examen de Selección Docente / 2009 },pdfauthor = {Luis Carrillo Gutiérrez} }

\renewcommand{\familydefault}{\sfdefault}

\begin{document}
\subsection*{Examen de Selección Docente}

\begin{itemize}
\item{Según \textit{Javier Pulgar Vidal}, en el Perú, existen ocho regiones naturales, siendo una de ellas la región \textbf{Yunga}, que significa :
\begin{tasks}
	\task{Valle intermedio - shalla.}
	\task{Valle cálido - mujer estéril.}
	\task{Valle bajo - amontonamiento.}
	\task{Valle seco - zona escarpada.}
	\task{Valle interandino - montón.}
\end{tasks}}
\item{El enunciado : ``El conocimiento reflexivo que tiene el hombre de su existencia y del mundo que lo rodea, basado en sus propias ideas y criterios'', está vinculado a :
\begin{tasks}(3)
\task{La conciencia.}
\task{El honor.}
\task{La vida.}
\task{La religión.}
\task{La identidad.}
\end{tasks}}
\item{Cuando arribaron los primeros exploradores españoles al altiplano, encontraron dos elementos fundamentales :
\begin{enumerate}[I. ]
\item{Riqueza ganadera.}
\item{Riqueza minera.}
\item{Tradiciones.}
\item{Chullpas.}
\item{Numerosa población.}
\end{enumerate}
\begin{tasks}
\task{III y IV.}
\task{II y III.}
\task{III y V.}
\task{I y V.}
\task{II y IV.}
\end{tasks}}
\item{En la región \textbf{Suní}, la vegetación espontánea es escasa, con pequeños bosquetes de árboles enanos y gramíneas poco significativos; pero existen algunas plantas que se han aclimatado, como :
\begin{tasks}
\task{Quisuar, Yareta y Sauco.}
\task{Colli, Ichu y Queñuas.}
\task{Molle, Lambram y Capullí.}
\task{Ciprés, Pino y Eucalipto.}
\task{Árboles frutales, Motuy y Cantuta.}
\end{tasks}}
\item{Los que narraron el proceso de conquista y descubrimiento se denominan cronistas \hspace{1cm} y los que contaron las incidencias de la conquista y colonización de América fueron los cronistas \hspace{1cm}.
\begin{tasks}
\task{menores y de la conquista.}
\task{mayores y del descubrimiento.}
\task{mayores y menores.}
\task{del incario y toledanos.}
\task{de las guerras civiles y pre-toledanos.}
\end{tasks}}
\item{Asumiendo que a finales de diciembre del año $2008$, el tipo de cambio fue de S/. $3.15$ por dólar, y a finales de enero del año $2009$ fue de S/. $3.20$ por dólar; entonces, durante el período considerado :
\begin{tasks}
\task{El dólar se ha apreciado en $0.05$\%.}
\task{El tipo de cambio se ha apreciado en $0.58$\%.}
\task{El Nuevo Sol se ha depreciado en $0.05$\%.}
\task{El tipo de cambio ha aumentado en S/. $1.58$.}
\task{El Nuevo Sol se ha depreciado en $1.58$\%.}
\end{tasks}}
\item{Las legislaturas ordinarias se llevan a cabo con el objeto de realizar un trabajo más efectivo. El congreso se reúne en dos legislaturas ordinarias :
\begin{enumerate}[I. ]
\item{La Primera Legislatura Ordinaria se realiza entre el 27 de julio y el 15 de diciembre.}
\item{La Segunda Legislatura Ordinaria se realiza entre el 01 de abril y el 31 de mayo.}
\item{La Primera Legislatura Ordinaria se realiza entre el 15 de agosto y el 31 de diciembre.}
\item{La Segunda Legislatura Ordinaria se realiza entre el 01 de marzo y el 27 de julio.}
\end{enumerate}
Señale la respuesta correcta :
\begin{tasks}
\task{II y III.}
\task{III y IV.}
\task{I y III.}
\task{IV y I.}
\task{I y II.}
\end{tasks}}
\item{Recién, Martha es capaz de comprender teorías acerca del origen del hombre. ¿En qué etapa se encuentra Martha?
\begin{tasks}
\task{Infancia.}
\task{Adultez.}
\task{Niñez.}
\task{Adolescencia.}
\task{Juventud.}
\end{tasks}}
\item{Los sitios arqueológicos de \textbf{Tanka Tanka} y \textbf{Checca}, pertenecieron al señorío altiplánico de :
\begin{tasks}
\task{Collas.}
\task{Pacajes.}
\task{Soras.}
\task{Lupacas.}
\task{Carís.}
\end{tasks}}
\item{Vienen a ser los movimientos horizontales y verticales (fruto del desplazamiento de las placas litosféricas), los mismos que están originados por : 
\begin{tasks}
\task{Las geodinámicas internas.}
\task{Las fuerzas mixtas.}
\task{Las corrientes marinas.}
\task{Las corrientes convectivas.}
\task{Las fuerzas internas.}
\end{tasks}}
\item{El proceso de emancipación del Perú significó, en el plano educativo y cultural, una oportunidad de renovación desde una educación colonialista y dogmática a una liberal. Este cambio se produjo con la creación de :
\begin{tasks}
\task{La Biblioteca Nacional, el Museo Nacional y la Escuela Normal.}
\task{La Biblioteca Nacional y la Biblioteca del Congreso.}
\task{La Biblioteca Nacional y nuevas escuelas en la Universidad Mayor de San Marcos.}
\task{El Museo Nacional y Escuela Nacional de Ingenieros.}
\task{La Escuela Normal.}
\end{tasks}}
\item{Las huacas del Sol y de la Luna son centros ceremoniales que corresponden a la cultura :
\begin{tasks}(3)
\task{Chimú.}
\task{Lupaca.}
\task{Kolla.}
\task{Mochica.}
\task{Lambayeque.}
\end{tasks}}
\item{En la gesta tupacamarista, el momento más importante de la historia de Puno, fue la :
\begin{tasks}
\task{La rebelión de Andrés Ingaricona.}
\task{La batalla de Quequerani.}
\task{La batalla de Condorcuyo.}
\task{La guerra de los Wiracochas.}
\task{La rebelión de Pedro Vilcapaza.}
\end{tasks}
}
\end{itemize}
\end{document}


